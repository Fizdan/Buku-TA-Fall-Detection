\chapter{PENGUJIAN DAN ANALISA}
\vspace{1ex}

\section*{}
Pada penelitian ini, dipaparkan hasil pengujian serta analisis dari desain sistem dan implementasi.  Data yang digunakan dalam pengujian data diambil dari IP Camera milik Dinas Perhubungan Kota Surabaya yang terpasang diseluruh lalu lintas Kota Surabaya. Pengambilan data video dilakukan di Surabaya Intelligent Transport System (SITS) dimana merupakan kantor Dinas Perhubungan khu- sus untuk pengendalian seluruh IP Camera lalu lintas yang tersebar di Kota Surabaya. Pengujian dilakukan dalam beberapa bagian se- bagai berikut:
\vspace{1ex}
\begin{enumerate}[nolistsep]
	\item Pengujian Peforma berdasarkan Lokasi
	\item Pengujian Peforma berdasarkan Kondisi.
	\item Pengujian Deteksi berdasarkan Objek Pelanggar.
	\item Pengujian Sistem
	
	\vspace{1ex}

\end{enumerate}
Dengan dilaksanakannya beberapa pengujian tersebut, sehingga dapat ditarik kesimpulan dari pelaksanaan tugas akhir ini.
\vspace{1ex}

\section{Pengujian Peforma berdasarkan Lokasi}
\vspace{1ex}

Pengujian peforma berdasarkan lokasi bertujuan untuk meng- etahui tingkat keakurasian pada YOLOv3 dan YOLOv3-tiny terha- dap lokasi yang memiliki karakteristik yang berbeda-beda. Pemilih- an lokasi didasarkan kondisi seperti sudut pandang kamera dan po- sisi kamera yang hampir dimiliki oleh semua kamera yang tersebar

\section{Pengujian Peforma berdasarkan Lokasi}
\vspace{1ex}

Pengujian peforma berdasarkan lokasi bertujuan untuk meng- etahui tingkat keakurasian pada YOLOv3 dan YOLOv3-tiny terha- dap lokasi yang memiliki karakteristik yang berbeda-beda. Pemilih- an lokasi didasarkan kondisi seperti sudut pandang kamera dan po- sisi kamera yang hampir dimiliki oleh semua kamera yang tersebar

\section{Pengujian Peforma berdasarkan Lokasi}
\vspace{1ex}

Pengujian peforma berdasarkan lokasi bertujuan untuk meng- etahui tingkat keakurasian pada YOLOv3 dan YOLOv3-tiny terha- dap lokasi yang memiliki karakteristik yang berbeda-beda. Pemilih- an lokasi didasarkan kondisi seperti sudut pandang kamera dan po- sisi kamera yang hampir dimiliki oleh semua kamera yang tersebar

\section{Pengujian Peforma berdasarkan Lokasi}
\vspace{1ex}

Pengujian peforma berdasarkan lokasi bertujuan untuk meng- etahui tingkat keakurasian pada YOLOv3 dan YOLOv3-tiny terha- dap lokasi yang memiliki karakteristik yang berbeda-beda. Pemilih- an lokasi didasarkan kondisi seperti sudut pandang kamera dan po- sisi kamera yang hampir dimiliki oleh semua kamera yang tersebar
