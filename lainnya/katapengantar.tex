\begin{center}
\Large\textbf{KATA PENGANTAR}
\end{center}
\vspace{1ex}

\setlength{\parindent}{0.9cm} Puji dan syukur kehadirat Tuhan Yang Maha Esa atas segala karunia-Nya, penulis  dapat menyelesaikan penelitian ini dengan judul \textbf{Pengembangan Simulator Mengemudi Untuk Riset Deteksi Pengemudi Mengantuk \textit{(Driving Simulator Development for Drowsy Driver Detection Research)}
}.
\vspace{1ex}

Penelitian ini disusun dalam rangka pemenuhan bidang riset di Departemen Teknik Komputer ITS, Bidang  Studi \textit{Game Technology}, serta digunakan sebagai persyaratan menyelesaikan pendidikan  S1. Oleh karena itu, penulis mengucapkan terima kasih kepada:
\vspace{1ex}

\begin{enumerate}[nolistsep]
  \item Bapak, ibu, adik, dan keluarga saya, atas semangat dan dukungan untuk tetap berkuliah
  \item Bapak Dr. Reza Fuad Rachmadi, S.T., M.T.
  \item Bapak Arief Kurniawan, ST., MT.
  \item Bapak - Ibu dosen pengajar Departemen Teknik Komputer, atas pengajaran, bimbingan, serta perhatian yang diberikan kepada penulis selama ini.
  \item Teman - teman Asisten Lab B201 Telematika yang selalu membantu dan menemani
  \item Teman - teman UKM Kendo ITS yang selalu memberi semangat
  \item Serta teman - teman angkatan 2017 yang telah bersama - sama melalui kehidupan perkuliahan bersama penulis
\end{enumerate}
\vspace{1ex}

Kesempurnaan hanya milik Allah SWT, untuk itu penulis memohon segenap kritik dan saran yang  membangun. Semoga penelitian ini dapat memberikan manfaat bagi kita semua. Amin.
\begin{flushright}
\begin{tabular}[b]{c}
  Surabaya, Juni 2021
  \\
  \\
  Penulis
\end{tabular}
\end{flushright}