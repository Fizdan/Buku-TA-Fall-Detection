\begin{center}
\Large\textbf{ABSTRAK}
\end{center}
\vspace{1ex}

\begin{adjustwidth}{-0.2cm}{}
\begin{tabular}{lcp{0.6\linewidth}}
Nama Mahasiswa &:& Hafizh Fauzan \\
Judul Tugas Akhir &:& Deteksi Jatuh pada Manusia Lanjut Usia Berbasis 3D - CNN pada Sistem Tertanam\\
Pembimbing &:& 1. Dr. Reza Fuad Rachmadi, S.T., M.T. \\
& & 2. Arief Kurniawan, ST., MT.\\
\end{tabular}
\end{adjustwidth}
\vspace{1ex}

\setlength{\parindent}{0cm} Jatuh adalah penanda kelemahan, imobilitas, dan gangguan kesehatan akut dan kronis pada orang tua. Bahkan ketika cederanya tidak begitu serius, lansia seringkali kesulitan untuk bangkit tanpa bantuan ,terkadang mengarah ke 'long-lie' di mana lansia tetap terjebak di lantai untuk periode waktu yang lama. 'Long-lie' dapat menyebabkan dehidrasi, \textit{Ulkus dekubitus}, pneumonia, hipotermia dan kematian. Pada Tugas Akhir ini akan dikembangkan sebuah sistem yang dapat mendeteksi jatuh pada manusia lanjut usia menggunakan algoritma \textit{3D Convolutional Neural Network} berbasis sistem tertanam. Adapun training data yang digunakan berasal dari beberapa dataset publik dan akan dibuat dataset pribadi untuk testing dan revisi sistem. Hasil yang diharapkan melalui Tugas Akhir ini adalah terciptanya sebuah sistem pendeteksi jatuh yang dapat melakukan kontak dengan keluarga dan rumah sakit jika diperlukan apabila terdeteksi jatuh sehingga penderita dapat mengalami penanganan medis dengan cepat.


\vspace{2ex}

Kata Kunci : Jatuh, Deteksi, Sistem
\newpage