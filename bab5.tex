\chapter{PENUTUP}
\vspace{1ex}

\section{Kesimpulan}
\vspace{1ex}

Dari hasil pengujian yang sudah dilakukan dapat ditarik beberapa kesimpulan sebagai berikut:
\vspace{1ex}

\begin{enumerate}[nolistsep]

\item Tombol - tombol jumlah lajur pada \textit{Interface - Main Menu} telah berkorelasi dengan benar terhadap \textit{scene} yang dimuat
\item Pengujian pengambilan data kecepatan menghasilkan data berdasarkan kalkulasi vektor global, diperlukan pengujian untuk memverifikasi keakuratan data tersebut.
\item Pengujian pengambilan data spasial menghasilkan data relatif posisi mobil terhadap garis pinggir jalan, dapat disimpulkan data tersebut dapat digunakan untuk suplemen data pengujian pengambilan data \textit{response time}, diperlukan pengujian untuk memverifikasi keakuratan data tersebut 
\item Pengujian citra webcam memiliki \textit{performance cost} yang sangat tinggi, yaitu \textit{execution time} tiap framenya mencapai 250-550 milisekon, hal ini disebabkan oleh proses unity dalam melakukan \textit{encoding} data berupa \textit{texture} menjadi suatu citra. Permasalahan ini ada pada level perangkat keras (GPU dan CPU). Diperlukannya suatu kompromi antara \textit{performance} dan akurasi
\item Proses kalkulasi data serta berjalannya \textit{script} utama pada \textit{unity} tidak terlalu berpengaruh terhadap respon \textit{steering wheel}, nilai error mendekati 0 persen atau akurat hingga 5 angka dibelakang koma (0.000001\%)(gambar \ref{fig:4.8}) hal ini disebabkan oleh kecilnya \textit{performance cost} dari \textit{script} tersebut (20-100 milisekon).
\item Pengujian UX tidak konklusif, yang disebabkan oleh situasi dan kondisi pandemi \textit{COVID-19}. Diperlukannya pengujian UX dengan jumlah responden yang lebih banyak sehingga dapat mewakili target demografi pengguna yang dituju.

\end{enumerate}
\vspace{1ex}

\section{Saran}
\vspace{1ex}

Untuk pengembangan penelitian selanjutnya terdapat beberapa saran sebagai berikut :
\vspace{1ex}

\begin{enumerate}[nolistsep]
	
	\item Melakukan \textit{refactor} / penataan ulang terhadap struktur source code.
	\vspace{1ex}
	
	\item Mengurangi \textit{performance cost} dari source code.
	\vspace{1ex}
	
	\item Meningkatkan estetik dari simulator mulai dari UI, kualitas objek 3D, serta animasi - animasi atau detail - detail  lain yang dapat meningkatkan imersifitas dari simulator.
	
	\item Menambah kapabilitas dari simulator dengan menambah jenis data yang bisa diambil oleh simulator.
	\vspace{1ex}

    \item Melakukan survey terhadap pengguna untuk fitur yang perlu ditambahkan pada simulator ini
	\vspace{1ex}
	
	\item Melakukan survey kuesioner dengan jumlah responden yang lebih banyak agar mewakili target demografi pengguna yang dituju
	\vspace{1ex}

\end{enumerate}