\documentclass[a5paper, 10pt, twoside, bahasa]{report}
\title{Tugas Akhir - Deteksi Jatuh}
\usepackage{graphicx}
\usepackage{hyphenat}
\usepackage{comment}
\usepackage{array}
\usepackage{multirow}
\usepackage{rotating}
\usepackage{booktabs}
\usepackage[ruled,lined,commentsnumbered,linesnumbered]{algorithm2e}
\usepackage{algpseudocode}
\usepackage{makeidx}
\makeindex
\usepackage[pdfauthor={Rohman Widiyanto},bookmarksnumbered,pdfborder={0 0 0}]{hyperref}  
\usepackage[indonesian]{babel}
\usepackage{epsfig}
\usepackage{subfig}
\usepackage[top=25mm,left=25mm,right=20mm,bottom=25mm]{geometry}
\usepackage{pdflscape}
\usepackage{setspace}  
\usepackage{type1cm}
\usepackage{lettrine}
\usepackage{hyperref}
\usepackage[pageref]{backref}
\usepackage{multirow}
\usepackage{fancyhdr} 		% Untuk pengaturan header dan footer yang lebih kompleks
\usepackage{etoolbox} 		% Untuk melakukan perubahan (patch) command internal LaTeX
\usepackage{url}
\usepackage{longtable}
\usepackage{float}
\floatstyle{boxed}
\newfloat{program}{thp}{lop}
\floatname{program}{Program}
\usepackage[fleqn]{amsmath}
\usepackage{enumitem}
\usepackage{nonfloat}
\usepackage{ulem}
\usepackage[final]{pdfpages}
\usepackage [raggedright]{titlesec}

\usepackage{array}
\usepackage{multicol}
\usepackage{listings}
\usepackage{wrapfig}

% Caption label bold
\usepackage[labelfont=bf]{caption}
\captionsetup{labelfont=bf}

% Jarak caption dengan obyek
\captionsetup[figure]{font=small,skip=5pt}
\captionsetup[table]{font=small,skip=5pt}
\captionsetup[lstlisting]{font=small,skip=5pt}

% Caption nama
\renewcommand{\figurename}{Gambar}
\renewcommand{\tablename}{Tabel}
\renewcommand{\lstlistingname}{Kode}

% Buat source code
\usepackage{courier}
\lstset{
	% language=C++, 						% Bahasa pengrograman yang digunakan
	basicstyle=\ttfamily \footnotesize,	% Jenis font dalam listing & Ukuran font
	% numbers=left, 						% Posisi angka untuk line-number
	% numberstyle=\footnotesize, 		% Ukuran angka untuk line-numbers
	% stepnumber=1, 						% Jarak setiap line-numbers
	% numbersep=5pt, 					% Ukuran line-numbers
	backgroundcolor=\color{white}, 		% Warna background. Gunakan \usepackage{color} dulu
	showspaces=false, 					% Show spaces adding particular underscores
	showstringspaces=false, 				% Underline spaces within strings
	showtabs=false, 						% Show tabs within strings adding particular underscores
	frame=single, 						% Tambahkan Frame
	framesep=0.1pt,						% Jarak frame dengan list content keseluruhan
	framexbottommargin=4pt,				% Jarak frame dengan list content bawah
  	framextopmargin=4pt,					% Jarak frame dengan list content atas
  	framexleftmargin=0pt,				% Jarak frame dengan list content kiri
  	framexrightmargin=0pt,				% Jarak frame dengan list content kanan
	tabsize=2, 							% Sets default tabsize to 2 spaces
	captionpos=b,						% Posisi caption
	breaklines=true, 					% Line breaking
	breakatwhitespace=false, 			% Sets if automatic breaks should only happen at whitespace
	escapeinside={\%*}{*)}				% if you want to add a comment within your code
}

% Untuk cek nomor halaman
\usepackage{changepage}
\usepackage{graphicx}

%Untuk check mark dan x-mark
\usepackage{amssymb}% http://ctan.org/pkg/amssymb
\usepackage{pifont}% http://ctan.org/pkg/pifont
\newcommand{\cmark}{\ding{51}}%
\newcommand{\xmark}{\ding{55}}%

\usepackage{lipsum}
\hyphenation{meng-gerak-kan mem-per-kenal-kan me-nger-ja-kan sa-ran seg-men be-ru-pa rasp-ber-ry meng-hu-bun-kan ter-sim-pan smart-phone me-nya-ma-kan sin-kro-ni-sa-si ke-ce-pa-tan di-hu-bung-kan sam-bu-ngan me-ru-pa-kan meng-gu-na-kan ber-da-sar-kan di-la-ku-kan di-gu-na-kan di-ban-ding-kan}

% Definisi untuk "halaman sengaja dikosongkan"
\def\kosong{
  \vspace*{\fill}
  \begin{center}\textit{Halaman ini sengaja dikosongkan}\end{center}
  \vfill
}
\patchcmd{\cleardoublepage}{\hbox{}}{\kosong}{}{}

% Tambahkan PDF atau Gambar
\newif\ifpdf
\ifx\pdfoutput\undefined
   \pdffalse
\else
   \pdfoutput=1
   \pdftrue
\fi
\ifpdf
   \usepackage{graphicx}
   \usepackage{epstopdf}
   \DeclareGraphicsRule{.eps}{pdf}{.pdf}{`epstopdf #1}
   \pdfcompresslevel=9
\else
   \usepackage{graphicx}
\fi

% utk itemize yg lebih rapat
\newenvironment{packed_enum}{
\begin{enumerate}[nolistsep]
  \setlength{\itemsep}{0pt}
  \setlength{\parskip}{0pt}
  \setlength{\parsep}{0pt}
}{\end{enumerate}}

% Untuk citation
\newcommand{\tab}[1]{\hspace{.2\textwidth}\rlap{#1}}
\renewcommand*{\backreflastsep}{, }
\renewcommand*{\backreftwosep}{, }
\renewcommand*{\backref}[1]{}
\renewcommand*{\backrefalt}[4]{
  \ifcase #1
    No citations.
  \or
    (Dikutip pada halaman #2).
  \else
    (Dikutip pada halaman #2).
  \fi
}

% Pengaturan penomoran halaman menggunakan package fancyhdr
\fancyhf{} 								% Mengosongkan header dan footer
\renewcommand{\headrulewidth}{0pt} 		% Menghapus garis horizontal pada header
\pagestyle{fancy} 						% Mengubah pagestyle dokumen menjadi fancy
\fancyfoot[CE,CO]{\thepage}				% Footer kanan pada hal. ganjil dan sebaliknya
\patchcmd{\chapter}{plain}{fancy}{}{} 	% Mengubah pagestyle pada chapter menjadi fancy
\patchcmd{\chapter}{empty}{plain}{}{}

% Pengaturan format Chapter dan Section
\titleformat{\chapter}[display]{\bfseries\Large}{BAB \centering\thechapter}{0ex}{\vspace{0ex}\centering}[\vspace{0ex}]
\titleformat{\section}{\bfseries\large}{\MakeUppercase{\thesection}}{1ex}{}
\titlespacing*{\chapter}{0pt}{-4ex}{0pt}
\titlespacing{\section}{0pt}{0pt}{0pt}

\begin{document}
\singlespacing

% Sampul luar
\input{sampul/sampul-luar.tex}
\cleardoublepage

% Nomor halaman pembuka dimulai dari sini
\pagenumbering{roman}

% Sampul dalam
\input{sampul/sampul-dalam.tex}
\cleardoublepage

% Sampul inggris
% \input{sampul/sampul-inggris.tex}
% \cleardoublepage

\includepdf[pages=-, offset=0 0]{file/Pengesahan.pdf}


% Abstrak bahasa indonesia
\addcontentsline{toc}{chapter}{Abstrak}
\begin{center}
\Large\textbf{ABSTRAK}
\end{center}
\vspace{1ex}

\begin{adjustwidth}{-0.2cm}{}
\begin{tabular}{lcp{0.6\linewidth}}
Nama Mahasiswa &:& Hafizh Fauzan \\
Judul Tugas Akhir &:& Deteksi Jatuh pada Manusia Lanjut Usia Berbasis 3D - CNN pada Sistem Tertanam\\
Pembimbing &:& 1. Dr. Reza Fuad Rachmadi, S.T., M.T. \\
& & 2. Arief Kurniawan, ST., MT.\\
\end{tabular}
\end{adjustwidth}
\vspace{1ex}

\setlength{\parindent}{0cm} Jatuh adalah penanda kelemahan, imobilitas, dan gangguan kesehatan akut dan kronis pada orang tua. Bahkan ketika cederanya tidak begitu serius, lansia seringkali kesulitan untuk bangkit tanpa bantuan ,terkadang mengarah ke 'long-lie' di mana lansia tetap terjebak di lantai untuk periode waktu yang lama. 'Long-lie' dapat menyebabkan dehidrasi, \textit{Ulkus dekubitus}, pneumonia, hipotermia dan kematian. Pada Tugas Akhir ini akan dikembangkan sebuah sistem yang dapat mendeteksi jatuh pada manusia lanjut usia menggunakan algoritma \textit{3D Convolutional Neural Network} berbasis sistem tertanam. Adapun training data yang digunakan berasal dari beberapa dataset publik dan akan dibuat dataset pribadi untuk testing dan revisi sistem. Hasil yang diharapkan melalui Tugas Akhir ini adalah terciptanya sebuah sistem pendeteksi jatuh yang dapat melakukan kontak dengan keluarga dan rumah sakit jika diperlukan apabila terdeteksi jatuh sehingga penderita dapat mengalami penanganan medis dengan cepat.


\vspace{2ex}

Kata Kunci : Jatuh, Deteksi, Sistem
\newpage
\cleardoublepage

% Abstrak bahasa ingris
%\addcontentsline{toc}{chapter}{Abstract}
%\begin{center}
\Large\textbf{ABSTRACT}
\end{center}
\vspace{1ex}

\begin{adjustwidth}{-0.2cm}{}
\begin{tabular}{lcp{0.6\linewidth}}
\textit{Name} &:& Nama\\
\textit{Title} &:& \textit{Title} \\
\textit{Advisors} &:& 1. Pembimbing1 \\
& & 2. Pembimbing2 \\
\end{tabular}
\end{adjustwidth}
\vspace{1ex}

	\setlength{\parindent}{0cm} Explain
	\vspace{2ex}
	
	\textit{Keywords : Train, Digital Map, Path Index, Speed Limit, Early Warning}
	\newpage

%\cleardoublepage

% Kata pengantar
\addcontentsline{toc}{chapter}{KATA PENGANTAR} % kata pengantar
\begin{center}
\Large\textbf{KATA PENGANTAR}
\end{center}
\vspace{1ex}

\setlength{\parindent}{0.9cm} Puji dan syukur kehadirat Tuhan Yang Maha Esa atas segala karunia-Nya, penulis  dapat menyelesaikan penelitian ini dengan judul \textbf{Deteksi Jatuh pada Manusia Lanjut Usia Berbasis 3D - CNN pada Sistem Tertanam \textit{(3D - CNN Fall Detection on Elderly Based on Embedded System)}
}.
\vspace{1ex}

Penelitian ini disusun dalam rangka pemenuhan bidang riset di Departemen Teknik Komputer ITS, Bidang  Studi \textit{Telematic}, serta digunakan sebagai persyaratan menyelesaikan pendidikan  S1. Oleh karena itu, penulis mengucapkan terima kasih kepada:
\vspace{1ex}

\begin{enumerate}[nolistsep]
  \item Bapak, ibu, adik, dan keluarga saya, atas semangat dan dukungan untuk tetap berkuliah
  \item Bapak Dr. Reza Fuad Rachmadi, S.T., M.T.
  \item Bapak Arief Kurniawan, ST., MT.
  \item Bapak - Ibu dosen pengajar Departemen Teknik Komputer, atas pengajaran, bimbingan, serta perhatian yang diberikan kepada penulis selama ini.
  \item Teman - teman Asisten Lab B201 Telematika yang selalu membantu dan menemani
  \item Teman - teman UKM Kendo ITS yang selalu memberi semangat
  \item Serta teman - teman angkatan 2017 yang telah bersama - sama melalui kehidupan perkuliahan bersama penulis
\end{enumerate}
\vspace{1ex}

Kesempurnaan hanya milik Allah SWT, untuk itu penulis memohon segenap kritik dan saran yang  membangun. Semoga penelitian ini dapat memberikan manfaat bagi kita semua. Amin.
\begin{flushright}
\begin{tabular}[b]{c}
  Surabaya, Juni 2021
  \\
  \\
  Penulis
\end{tabular}
\end{flushright}
\cleardoublepage

% Daftar isi
\renewcommand*\contentsname{DAFTAR ISI}
\addcontentsline{toc}{chapter}{\contentsname}
\titlespacing*{\chapter}{0pt}{-4ex}{2ex}
\tableofcontents	
\cleardoublepage

% Daftar gambar
\renewcommand*\listfigurename{DAFTAR GAMBAR}
\addcontentsline{toc}{chapter}{\listfigurename}
\titlespacing*{\chapter}{0pt}{-4ex}{2ex}
\listoffigures
\cleardoublepage

% Daftar tabel
\renewcommand*\listtablename{DAFTAR TABEL}
\addcontentsline{toc}{chapter}{\listtablename}
\titlespacing*{\chapter}{0pt}{-4ex}{2ex}
\listoftables
\cleardoublepage

% Nomenklatur
\addcontentsline{toc}{chapter}{NOMENKLATUR}
\begin{center}
	\Large\textbf{NOMENKLATUR}
\end{center}
\vspace{1ex}

\begin{tabular}{c m{30em}}
	$fps$	& : \textit{Frame Per Second} / Jumlah Citra Perdetik\\
	$unit$	& : unit pengukuran \textit{Unity Game Engine} \\
	
	
\end{tabular}
\vspace{1ex}
\cleardoublepage

% BAB isi buku
\titleformat{\chapter}[display]{\bfseries\Large}{BAB \centering\thechapter}{0ex}{\vspace{0ex}\centering}[\vspace{0ex}]
\titleformat{\section}{\bfseries\large}{\MakeUppercase{\thesection}}{1ex}{}
\titleformat{\subsection}{\bfseries\large}{\MakeUppercase{\thesubsection}}{1ex}{}
\titleformat{\subsubsection}{\bfseries\large}{\MakeUppercase{\thesubsubsection}}{1ex}{}
\titlespacing*{\chapter}{0pt}{-4ex}{0pt}
\titlespacing{\section}{0pt}{0pt}{0pt}
\titlespacing{\subsection}{0pt}{0pt}{0pt}
\titlespacing{\subsubsection}{0pt}{0pt}{0pt}

% Indent paragraph
\setlength{\parindent}{0.8cm}

% Penambahan halaman kosong otomatis
\chapter{PENDAHULUAN}
\pagenumbering{arabic}
\vspace{1ex}

\section*{}
Penelitian ini di latar belakangi oleh berbagai kondisi yang menjadi acuan. Selain itu juga terdapat beberapa permasalahan yang akan dijawab sebagai luaran dari penelitian.
\vspace{1ex}

\section{Latar belakang}
\vspace{1ex}

Indonesia mulai memasuki periode aging population, dimana terjadi peningkatan umur harapan hidup yang diikuti dengan peningkatan jumlah lansia, dan Indonesia mengalami peningkatan jumlah penduduk lansia dari 18 juta jiwa (7,56\%) pada tahun 2010, menjadi 25,9 juta jiwa (9,7\%) pada tahun 2019, dan diperkirakan akan terus meningkat dimana tahun 2035 menjadi 48,2 juta jiwa (15,77\%) \cite{cit:1}. Berdasarkan data dari Badan Pusat Statistika (BPS) Indonesia keberadaan lansia yang tinggal sendiri, di mana persentasenya mencapai 9,38 \%. Jika dilihat berdasarkan tipe daerah, persentase lansia di perdesaan yang tinggal sendiri lebih tinggi dibandingkan lansia di perkotaan (10,10 \% berbanding 8,74 \%). Bahkan, terdapat kesenjangan yang cukup tinggi pada lansia yang tinggal sendiri antara lansia perempuan dengan lakilaki (13,39 \% berbanding 4,98 \%) \cite{cit:2}.Lansia yang tinggal sendiri digambarkan sebagai kelompok yang berisiko dan membutuhkan perhatian khusus \cite{cit:4}. 
\vspace{1ex}

Menurut data World Health Organization (WHO) pada tahun 2018, Jatuh adalah penyebab utama kedua kematian akibat cedera yang tidak disengaja atau tidak disengaja di seluruh dunia \cite{cit:5}. Berdasarkan data dari Centers for Disease Control and Prevention tingkat kematian akibat jatuh yang disesuaikan dengan usia adalah 64 kematian per 100.000 orang dewasa yang lebih tua, tingkat kematian akibat jatuh di antara orang dewasa berusia 65 tahun ke atas meningkat sekitar 30\% dari 2009 hingga 2018. \cite{cit:6}. Jatuh adalah penanda kelemahan, imobilitas, dan gangguan kesehatan akut dan kronis pada orang tua. Jatuh pada gilirannya mengurangi fungsinya dengan menyebabkan cedera, keterbatasan aktivitas, takut jatuh, dan kehilangan mobilitas. Kebanyakan cedera pada lansia adalah akibat jatuh; patah tulang pinggul, lengan bawah, humerus, dan panggul biasanya diakibatkan oleh efek gabungan dari jatuh dan osteoporosis \cite{cit:7}. Bahkan ketika cederanya tidak begitu serius, lansia sering kali kesulitan untuk bangkit tanpa bantuan \cite{cit:8},terkadang mengarah ke 'long-lie' di mana lansia tetap terjebak di lantai untuk waktu periode waktu yang lama. 'long-lie' dapat menyebabkan dehidrasi, \textit{Ulkus dekubitus}, pneumonia, hipotermia dan kematian \cite{cit:9}.
 \vspace{1ex} 

Teknologi untuk melakukan deteksi jatuh sudah ada, dan umumnya dapat dikategorikan menjadi 2 jenis, yaitu secara visual dan secara \textit{wearable}. Tetapi, lansia memiliki kecenderungan akan membawa hal penting sehingga deteksi secara \textit{wearable} tidak disarankan dikarenakan lansia harus selalu menggunakan alat tersebut yang juga membuat lansia tidak nyaman. Mayoritas lansia juga memiliki kondisi keuangan yang tidak kuat dikarenakan bergantung pada uang pensiun dan tidak memiliki pemasukan.
\vspace{1ex} 

Oleh karena itu, pada tugas akhir ini akan dikembangkan sebuah sistem untuk melakukan deteksi jatuh dan pelaporan berbasis visual dan juga sistem tertanam dengan harga terjangkau. Metode yang digunakan, yaitu 3D-CNN, berfungsi untuk melakukan deteksi jatuh yang melakukan deteksi secara \textit{frame sequence}, sehingga meningkatkan akurasi dan mengurangi \textit{false alarm}. Diharapkan dengan pengembangan tugas akhir ini, sistem dapat melakukan deteksi jatuh pada lansia sehingga dapat menghubungi keluarga terdekat jika terjadi kejadian jatuh. 
\vspace{1ex} 

\section{Permasalahan}
\vspace{1ex}

Berdasarkan data yang telah dipaparkan di latar belakang, dapat dirumuskan beberapa rumusan masalah sebagai berikut yaitu meningkatnya angka manusia lanjut usia serta jumlah manusia lanjut usia tinggal sendiri yang relatif banyak, dimana sulit mendapatkan bantuan. Lalu risiko jatuh pada manusia lanjut usia semakin meningkat dengan bertambahnya umur, dikarenakan penurunan fisik. Kemudian manusia lanjut usia mengalami  penurunan  fisik, sehingga  jika  mengalami  kecelakaan,  luka  yang diderita dapat menyebabkan kematian jika tidak cepat ditangani.
\vspace{1ex}

\section{Tujuan}
\vspace{1ex}

Penelitian ini bertujuan untuk merancang sebuah perangkat yang mampu untuk mendeteksi orang yang jatuh secara otomatis. Perangkat tersebut akan menggunakan input video dari \textit{IP Camera} yang diakses oleh sebuah \textit{Single Board Computer} yang didalamnya telah ditanamkan sebuah program pengolahan citra digital. Jika terdeteksi orang jatuh maka gambar akan dikirimkan kepada rumah sakit terdekat dan meminta bantuan. Alarm juga akan menyala untuk mencari orang terdekat dan memberitahukan keluarga baik secara SMS ataupun media sosial.
\vspace{1ex}

\section{Batasan masalah}
\vspace{1ex}
Batasan masalah yang timbul dari permasalahan Tugas Akhir ini adalah:
\vspace{1ex}
\begin{enumerate}[nolistsep]
    \item Pengujian dilakukan dalam ruangan
	\cite{cit:4}
	\vspace{1ex}

	\item Orang yang akan jatuh hanya 1 orang
	\vspace{1ex}
	
	\item Kegiatan Uji adalah kegiatan pengambilan data berupa :
	    \begin{enumerate}
	        \item Korelasi User Interface dengan Lajur yang dimuat.
            \item Kecepatan Mobil
            \item Informasi Spasial Mobil
            \item Respon Waktu Pengendara \textit{(Response Time)}
            \item Citra Wajah Pengendara
            \item \textit{Serial Data} dari \textit{Microcontroller}
            \item Respon Sinyal dari \textit{Steering Wheel Controller} terhadap simulator
            \item Kuesioner \textit{User Experience} / UX Pengguna
	    \end{enumerate}
	\vspace{1ex}
	
	\item Kegiatan Uji Menggunakan dataset pribadi
	\vspace{1ex}
		 
\end{enumerate}
\vspace{1ex}

\section{Sistematika Penulisan}
\vspace{1ex}
Laporan penelitian Tugas akhir ini tersusun dalam sistematika dan terstruktur sehingga mudah dipahami dan dipelajari oleh pembaca maupun seseorang yang ingin melanjutkan penelitian ini. Alur sistematika penulisan laporan penelitian ini yaitu:
\vspace{1ex}

\begin{enumerate}[nolistsep]
	\item BAB I Pendahuluan

	Bab ini berisi uraian tentang latar belakang permasalahan, penegasan dan alasan pemilihan judul, sistematika laporan, tujuan dan metodologi penelitian.
	\vspace{1ex}

	\item BAB II Dasar Teori

	Pada bab ini berisi tentang uraian secara sistematis teori-teori yang berhubungan dengan permasalahan yang dibahas pada penelitian ini. Teori-teori ini digunakan sebagai dasar dalam penelitian, yaitu sistem simulator dan pengambilan data variabel - variabel uji.
	\vspace{1ex}

	\item BAB III Perancangan Sistem dan Impementasi

	Bab ini berisi tentang penjelasan-penjelasan terkait eksperimen yang akan dilakukan dan langkah-langkah pengolahan data hingga menghasilkan visualisasi. Guna mendukung eksperimen pada penelitian ini, digunakanlah blok diagram atau \textit{work flow} agar penjelasan sistem yang akan dibuat dapat terlihat dan mudah dibaca untuk implementasi pada pelaksanaan tugas akhir.
	\vspace{1ex}

	\item BAB IV Pengujian dan Analisa

	Bab ini menjelaskan tentang pengujian eksperimen yang dilakukan terhadap data dan analisanya. Beberapa teknik visualisasi akan ditunjukan hasilnya pada bab ini dan dilakukan analisa terhadap hasil visualisasi dan informasi yang didapat dari hasil mengamati visualisasi yang tersaji
	\vspace{1ex}

	\item BAB V Penutup

	Bab ini merupakan penutup yang berisi kesimpulan yang diambil dari penelitian dan pengujian yang telah dilakukan. Saran dan kritik yang membangun untuk pengembangkan lebih lanjut juga dituliskan pada bab ini.
\end{enumerate}
\vspace{1ex}

\section{Relevansi}
\begin{enumerate}
    \item Validasi Skala Kantuk Karolinska \mbox{\textit{(Karolinska Sleepiness Scale)}} / KSS dengan Variabel – Variabel EEG
    \par
    Peneliti pada riset tersebut bertujuan untuk melakukan validasi terhadap skala tingkat kantuk yang di proposisikan oleh Institut Karolinska Swedia, dengan variabel – variabel EEG. Peneliti memiliki landasan riset data korelasi antara variabel – variabel EEG  yang sudah valid dan terbukti. Maka selanjutnya peneliti ingin melakukan analisa korelasi antara variabel – variabel EEG dengan tingkat kantuk yang dimiliki oleh subjek riset. Kesimpulan dari riset tersebut menyatakan, skala tingkat kantuk karolinska memiliki korelasi yang kuat dengan variabel – variabel EEG \cite{cit:6}
    \vspace{1ex}
    
    \item Tingkat Kantuk Subjektif,  Simulasi kinerja Mengemudi menggunakan durasi kedipan mata.
    \par
    Riset ini memiliki tujuan untuk memproposisikan Tingkat Kantuk secara subjektif, data yang digunakan pada paper riset ini berupa lama durasi tingkat kedipan mata, dari banyak data yang diambil pada riset ini didapatkan korelasi antara durasi kedip mata dengan tingkat kantuk, namun kemudian peneliti menyimpulkan bahwa untuk dapat menentukan pengukuran Tingkat Kantuk secara Subjektif, harus dilakukan replikasi riset ini beberapa kali lagi \cite{cit:7}
    \vspace{1ex}
    
    \item Persepsi – Waktu Respon terhadap Bahaya tak terduga dijalan
    \par
    \textit{Research Question} yang diangkat oleh peneliti pada riset ini adalah “Seberapa lama toleransi waktu yang diperbolehkan untuk pengemudi bereaksi terhadap bahaya tak terduga dijalan?”, atau pada riset ini disebut dengan Perception Time (PR). Peneliti juga menyebutkan bahwa terlalu tinggi menilai angka PR dapat meningkatkan biaya konstruksi jalan. Sedangkan terlalu rendah menilai angka PR dapat menyebabkan meningkatnya bahaya dijalan bagi pengendara kendaraan bermotor di jalan. Setelah melakukan penelitian, peneliti pada riset ini menyimpulkan bahwa waktu response yang ideal bagi 95\% populasi yang mengemudi adalah 1,6 detik. Peneliti menambahkan, kesimpulan tersebut hanya berlaku pada situasi yang sama dengan kondisi pengujian. Apabila terdapat bahaya yang lebih mengintimidasi, hal tersebut dapat menghasilkan kemungkinan nilai PR yang berbeda. \cite{cit:8}
    \vspace{1ex}11

\end{enumerate}
\vspace{1ex}
\cleardoublepage
\chapter{TINJAUAN PUSTAKA}
\vspace{1ex}

\section*{}
Demi mendukung penelitian ini, dibutuhkan beberapa teori penunjang sebagai bahan acuan dan referensi. Dengan demikian penelitian ini menjadi lebih terarah. 
\vspace{1ex}

\section{Jatuh pada Manusia}
\vspace{1ex}

Jatuh adalah kejadian yang tidak disadari dimana seseorang 
terjatuh dari tempat yang lebih tinggi ke tempat yang lebih 
rendah. Banyak sekali penyebab untuk jatuh, salah satunya 
adalah penurunan fisik pada majoritas lanjut usia. 
\vspace{1ex}

\section{Machine Learning}
\vspace{1ex}

Machine Learning (ML) atau Pembelajaran Mesin merupakan bagian 
dari Artificial Intelligence (AI) yang bertujuan untuk memberi 
optimalisasi dalam kriteria dengan cara menganalisa sampel data 
yang terdahulu yang sudah disimpan atau direkam untuk menghasilkan 
sebuah prediksi. Sehingga manusia tidak perlu mengindentifikasi 
sebuah proses sepenuhnya, karena dengan Machine Learning, komputer 
mampu membuat pola untuk membuat keputusan. Machine Learning 
melakukan training yang merupakan proses pembelajaran terhadap 
model data yang sudah terdefinisikan ke beberapa parameter 
(data training) yang menghasilkan beberapa pola sehingga komputer 
dapat melakukan proses klasifikasi berdasarkan pola atau ciri-ciri 
yang sudah didapatkan dalam proses training. Kemudian komputer 
dapat memberikan sebuah prediksi pada data baru selanjutnya 
berdasarkan hasil training. Machine Learning dapat memberi solusi 
dalam berbagai permasalahan seperti Vision (Visi Komputer), Speech 
Recognition (Pengenalan Suara) dan Robotics (Robotika).
\vspace{1ex}

\subsection{\textit{Supervised Learning}}
Desain sistem secara umum pada gambar \ref{fig: 3_1}, yang mencakup disiplin ilmu perangkat keras atau \textit{hardware}, ialah pengolahan citra gambar, visi komputer, sistem tertanam, serta pengolahan sinyal. Disiplin pengolahan citra gambar didapatkan dari pengambilan citra pengemudi menggunakan kamera, visi komputer didapatkan dari proses \textit{recognition} wajah pengemudi, sistem tertanam atau \textit{embedded system} didapatkan dari komunikasi data - data serial menggunakan Arduino atau Mikroprosesor yang tersambung dengan simulator, kemudian yang terakhir Pengolahan sinyal didapatkan dari pengolahan data - data analog seperti data \textit{Electroencephalography (EEG)} / detak jantung pengemudi, atau data \textit{Electrooculography (EOG)} / data kedipan mata dari pengemudi.
\vspace{1ex}

\subsection{\textit{Unsupervised Learning}}
Cakupan Tugas Akhir ini pada bagian perangkat lunak atau \textit{software}, lebih ditekankan dalam upaya pembuatan \textit{Simulation Environment} atau Lingkungan Simulasi menggunakan \textit{Unity Game Engine}. \textit{Simulation Environment} dibuat dengan menggunakan disiplin ilmu grafika komputer 3d (3d \textit{Computer Graphics}), serta juga menggunakan proses - proses pengembangan dari \textit{game engine} dan \textit{physics engine} yang lain.
\vspace{1ex}

\subsection{\textit{Reinforcement Learning}}
\textit{Output} atau keluaran yang diharapkan dari tugas akhir ini ialah, dihasilkannya suatu modul simulasi yang terintegrasi lengkap dengan \textit{tools - tools} dan \textit{peripheral} yang yang dapat mensimulasikan suatu pengalaman mengemudi menggunakan suatu \textit{simulator}, serta dapat melakukan proses pengambilan data - data primer yang valid, sehingga dapat diolah untuk proses riset selanjutnya.
\vspace{1ex}

\section{Deep Learning}
\vspace{1ex}

\begin{figure} [H]
	\captionsetup{justification=centering}
	\includegraphics[scale=0.2]{img/contoh-kurva-bezier.JPG}
	\caption{Contoh kurva bezier pada bidang 3 dimensi}
	\label{fig:2.1}
\end{figure}

Deep Learning merupakan artificial neural network yang memiliki banyak layer dan synapse weight. Deep learning dapat menemukan relasi tersembunyi atau pola yang rumit antara input dan
output, yang tidak dapat diselesaikan menggunakan multilayer perceptron (3 layers). Keuntungan utama deep learning yaitu mampu
merubah data dari non-linearly separable menjadi linearly separable
melalui serangkaian transformasi (hidden layers). Selain itu, deep
learning juga mampu mencari decision boundary yang berbentuk
non-linier, serta mengsimulasikan interaksi non-linier antar fitur.
Jadi, input ditransformasikan secara non-linier sampai akhirnya pada output, berbentuk distribusi class-assignment. Pada training
\vspace{1ex}

\section{3D Convolutional Neural Network (3D-CNN)}
\vspace{1ex}

\begin{figure} [H]
	\captionsetup{justification=centering}
	\includegraphics[scale=0.2]{img/contoh-kurva-bezier.JPG}
	\caption{Contoh kurva bezier pada bidang 3 dimensi}
	\label{fig:2.1}
\end{figure}

3D Convolutional Neural Network (CNN) merupakan cabang dari Multilayer Perceptron (MLP) yang digunakan untuk mengolah
data dua dimensi. CNN memiliki kedalaman jaringan yang tinggi
sehingga CNN termasuk dalam jenis Deep Neural Network. Perbedaan CNN dengan MLP terdapat pada neuron dimana pada MLP
setiap neuron hanya berukuran satu dimensi, sedangkan CNN setiap neuronnya berukuran dua dimensi. Pada CNN, operasi linier
menggunakan operasi konvolusi. Bobot pada CNN berbentuk empat dimensi seperti pada Gambar 2.2. Persamaan 2.1 untuk dimensi
bobot pada CNN.
\vspace{1ex}

\subsection{\textit{Convolutional Layer}}
Desain sistem secara umum pada gambar \ref{fig: 3_1}, yang mencakup disiplin ilmu perangkat keras atau \textit{hardware}, ialah pengolahan citra gambar, visi komputer, sistem tertanam, serta pengolahan sinyal. Disiplin pengolahan citra gambar didapatkan dari pengambilan citra pengemudi menggunakan kamera, visi komputer didapatkan dari proses \textit{recognition} wajah pengemudi, sistem tertanam atau \textit{embedded system} didapatkan dari komunikasi data - data serial menggunakan Arduino atau Mikroprosesor yang tersambung dengan simulator, kemudian yang terakhir Pengolahan sinyal didapatkan dari pengolahan data - data analog seperti data \textit{Electroencephalography (EEG)} / detak jantung pengemudi, atau data \textit{Electrooculography (EOG)} / data kedipan mata dari pengemudi.
\vspace{1ex}

\subsection{\textit{Subsampling Layer}}
Cakupan Tugas Akhir ini pada bagian perangkat lunak atau \textit{software}, lebih ditekankan dalam upaya pembuatan \textit{Simulation Environment} atau Lingkungan Simulasi menggunakan \textit{Unity Game Engine}. \textit{Simulation Environment} dibuat dengan menggunakan disiplin ilmu grafika komputer 3d (3d \textit{Computer Graphics}), serta juga menggunakan proses - proses pengembangan dari \textit{game engine} dan \textit{physics engine} yang lain.
\vspace{1ex}

\subsection{\textit{Batch Normalization Layer}}
Cakupan Tugas Akhir ini pada bagian perangkat lunak atau \textit{software}, lebih ditekankan dalam upaya pembuatan \textit{Simulation Environment} atau Lingkungan Simulasi menggunakan \textit{Unity Game Engine}. \textit{Simulation Environment} dibuat dengan menggunakan disiplin ilmu grafika komputer 3d (3d \textit{Computer Graphics}), serta juga menggunakan proses - proses pengembangan dari \textit{game engine} dan \textit{physics engine} yang lain.
\vspace{1ex}

\subsection{\textit{Fully-Connected Layer}}
\textit{Output} atau keluaran yang diharapkan dari tugas akhir ini ialah, dihasilkannya suatu modul simulasi yang terintegrasi lengkap dengan \textit{tools - tools} dan \textit{peripheral} yang yang dapat mensimulasikan suatu pengalaman mengemudi menggunakan suatu \textit{simulator}, serta dapat melakukan proses pengambilan data - data primer yang valid, sehingga dapat diolah untuk proses riset selanjutnya.
\vspace{1ex}

\section{Visi Komputer}
\vspace{1ex}

Visi Komputer adalah cabang Artificial Intelligent (AI) yang
mencakup proses analisa citra dan video. Visi komputer mengimplementasikan beberapa kemampuan visual manusia yang diteruskan
menuju otak seperti deteksi benda, pengenalan wajah dan mengenali bahaya.
Pada visi komputer, Deep Learning sering digunakan untuk pengenalan dan deteksi objek. Proses Deep Learning pada visi komputer memanfaatkan piksel pada citra untuk ekstrasi pola atau atribut
dari citra yang ingin dideteksi. Akan tetapi, hal tersebut mengakibatkan sistem komputasi menjadi lama karena pada suatu citra
mengandung ribuan piksel. Sehingga banyak arsitektur visi komputer membuat standar ukuran, jadi citra tersebut harus dipotong
atau diperkecil untuk mempercepat proses komputasi.
\vspace{1ex}

\section{Image Processing}
\vspace{1ex}

\begin{figure} [H]
	\captionsetup{justification=centering}
	\includegraphics[scale=0.2]{img/contoh-kurva-bezier.JPG}
	\caption{Contoh kurva bezier pada bidang 3 dimensi}
	\label{fig:2.1}
\end{figure}

Image Processing atau Pengolahan Citra merupakan teknik
dalam pemrosesan gambar dengan input berupa citra dua dimensi yang bertujuan untuk menyempurnakan citra atau mendapatkan
informasi yang berguna untuk diolah menjadi beberapa keputusan. Dalam operasi pemrosesan citra, operasi yang sering dilakukan
dalam gambar grayscale. Gambar grayscale didapatkan dari pemrosesan gambar berwarna yang didekomposisi menjadi komponen
merah (R), hijau (G) dan biru (B) yang diproses secara independen
sebagai gambar grayscale. Image Processing terbagi menjadi dalam
\vspace{1ex}

\section{Sistem Tertanam}
\vspace{1ex}

\begin{figure} [H]
	\captionsetup{justification=centering}
	\includegraphics[scale=0.2]{img/contoh-kurva-bezier.JPG}
	\caption{Contoh kurva bezier pada bidang 3 dimensi}
	\label{fig:2.1}
\end{figure}

Image Processing atau Pengolahan Citra merupakan teknik
dalam pemrosesan gambar dengan input berupa citra dua dimensi yang bertujuan untuk menyempurnakan citra atau mendapatkan
informasi yang berguna untuk diolah menjadi beberapa keputusan. Dalam operasi pemrosesan citra, operasi yang sering dilakukan
dalam gambar grayscale. Gambar grayscale didapatkan dari pemrosesan gambar berwarna yang didekomposisi menjadi komponen
merah (R), hijau (G) dan biru (B) yang diproses secara independen
sebagai gambar grayscale. Image Processing terbagi menjadi dalam.
\vspace{1ex}

\section{Metode Pengujian}
\vspace{1ex}

\begin{figure} [H]
	\captionsetup{justification=centering}
	\includegraphics[scale=0.2]{img/contoh-kurva-bezier.JPG}
	\caption{Contoh kurva bezier pada bidang 3 dimensi}
	\label{fig:2.1}
\end{figure}

\textit{3D Convolutional Neural Network} adalah suatu arsitektur neural network dimana mengekstrak fitur dari dimensi spasial dan temporal dengan melakukan konvolusi 3D, sehingga menangkap informasi gerakan yang dikodekan dalam beberapa bingkai yang berdekatan. Arsitektur yang digunakan menghasilkan banyak saluran informasi dari bingkai masukan, dan representasi fitur akhir menggabungkan informasi dari semua saluran.
\vspace{1ex}

\subsection{\textit{Recall}}
Desain sistem secara umum pada gambar \ref{fig: 3_1}, yang mencakup disiplin ilmu perangkat keras atau \textit{hardware}, ialah pengolahan citra gambar, visi komputer, sistem tertanam, serta pengolahan sinyal. Disiplin pengolahan citra gambar didapatkan dari pengambilan citra pengemudi menggunakan kamera, visi komputer didapatkan dari proses \textit{recognition} wajah pengemudi, sistem tertanam atau \textit{embedded system} didapatkan dari komunikasi data - data serial menggunakan Arduino atau Mikroprosesor yang tersambung dengan simulator, kemudian yang terakhir Pengolahan sinyal didapatkan dari pengolahan data - data analog seperti data \textit{Electroencephalography (EEG)} / detak jantung pengemudi, atau data \textit{Electrooculography (EOG)} / data kedipan mata dari pengemudi.
\vspace{1ex}

\subsection{\textit{Precision}}
Cakupan Tugas Akhir ini pada bagian perangkat lunak atau \textit{software}, lebih ditekankan dalam upaya pembuatan \textit{Simulation Environment} atau Lingkungan Simulasi menggunakan \textit{Unity Game Engine}. \textit{Simulation Environment} dibuat dengan menggunakan disiplin ilmu grafika komputer 3d (3d \textit{Computer Graphics}), serta juga menggunakan proses - proses pengembangan dari \textit{game engine} dan \textit{physics engine} yang lain.
\vspace{1ex}

\subsection{\textit{F-Measure}}
Cakupan Tugas Akhir ini pada bagian perangkat lunak atau \textit{software}, lebih ditekankan dalam upaya pembuatan \textit{Simulation Environment} atau Lingkungan Simulasi menggunakan \textit{Unity Game Engine}. \textit{Simulation Environment} dibuat dengan menggunakan disiplin ilmu grafika komputer 3d (3d \textit{Computer Graphics}), serta juga menggunakan proses - proses pengembangan dari \textit{game engine} dan \textit{physics engine} yang lain.
\vspace{1ex}

\subsection{\textit{mean Average Precision (mAP)}}
\textit{Output} atau keluaran yang diharapkan dari tugas akhir ini ialah, dihasilkannya suatu modul simulasi yang terintegrasi lengkap dengan \textit{tools - tools} dan \textit{peripheral} yang yang dapat mensimulasikan suatu pengalaman mengemudi menggunakan suatu \textit{simulator}, serta dapat melakukan proses pengambilan data - data primer yang valid, sehingga dapat diolah untuk proses riset selanjutnya.
\vspace{1ex}

\subsection{\textit{Confusion Matrix}}
\textit{Output} atau keluaran yang diharapkan dari tugas akhir ini ialah, dihasilkannya suatu modul simulasi yang terintegrasi lengkap dengan \textit{tools - tools} dan \textit{peripheral} yang yang dapat mensimulasikan suatu pengalaman mengemudi menggunakan suatu \textit{simulator}, serta dapat melakukan proses pengambilan data - data primer yang valid, sehingga dapat diolah untuk proses riset selanjutnya.
\vspace{1ex}
\cleardoublepage
\chapter{DESAIN DAN IMPLEMENTASI SISTEM}
\vspace{1ex}

\section*{}
	Penelitian ini dilaksanakan sesuai dengan desain sistem berikut dengan implementasinya. Desain sistem merupakan konsep dari pembuatan dan perancangan infrastruktur kemudian diwujudkan dalam bentuk blok-blok alur yang harus dikerjakan. Pada bagian implementasi merupakan pelaksanaan teknis untuk setiap blok pada desain sistem.
\vspace{1ex}

\section{Cakupan Tugas Akhir}
\vspace{1ex}

	Tugas akhir ini merupakan salah satu bentuk implementasi grafika komputer untuk mensimulasikan pengalaman berkendara yang digabungkan dengan sistem \textit{Microcontroller} untuk pengambilan data, berikut pada Gambar 3.1 adalah cakupan Tugas Akhir dari Desain Sistem.
\begin{figure}  [!htb]
	\captionsetup{justification=centering}
	\includegraphics[scale=0.55]{img/cakupanTA.JPG}
	\caption{Blok Diagram Cakupan Disiplin Ilmu Tugas Akhir}
	\label{fig: 3_1}
\end{figure}
\vspace{1ex}
    \subsection{Cakupan \textit{Hardware}}
    Desain sistem secara umum pada gambar \ref{fig: 3_1}, yang mencakup disiplin ilmu perangkat keras atau \textit{hardware}, ialah pengolahan citra gambar, visi komputer, sistem tertanam, serta pengolahan sinyal. Disiplin pengolahan citra gambar didapatkan dari pengambilan citra pengemudi menggunakan kamera, visi komputer didapatkan dari proses \textit{recognition} wajah pengemudi, sistem tertanam atau \textit{embedded system} didapatkan dari komunikasi data - data serial menggunakan Arduino atau Mikroprosesor yang tersambung dengan simulator, kemudian yang terakhir Pengolahan sinyal didapatkan dari pengolahan data - data analog seperti data \textit{Electroencephalography (EEG)} / detak jantung pengemudi, atau data \textit{Electrooculography (EOG)} / data kedipan mata dari pengemudi.
    
    \subsection{Cakupan \textit{Software}}
   Cakupan Tugas Akhir ini pada bagian perangkat lunak atau \textit{software}, lebih ditekankan dalam upaya pembuatan \textit{Simulation Environment} atau Lingkungan Simulasi menggunakan \textit{Unity Game Engine}. \textit{Simulation Environment} dibuat dengan menggunakan disiplin ilmu grafika komputer 3d (3d \textit{Computer Graphics}), serta juga menggunakan proses - proses pengembangan dari \textit{game engine} dan \textit{physics engine} yang lain.
    
    \subsection{\textit{Output yang diharapkan}}
    \textit{Output} atau keluaran yang diharapkan dari tugas akhir ini ialah, dihasilkannya suatu modul simulasi yang terintegrasi lengkap dengan \textit{tools - tools} dan \textit{peripheral} yang yang dapat mensimulasikan suatu pengalaman mengemudi menggunakan suatu \textit{simulator}, serta dapat melakukan proses pengambilan data - data primer yang valid, sehingga dapat diolah untuk proses riset selanjutnya.


\section{Desain Sistem}
\vspace{1ex}
	Pada Tugas Akhir ini, dilakukan penggabungan perangkat lunak berupa \textit{game engine} untuk mensimulasikan proses berkendara, dengan perangkat keras berupa \textit{controller} dari \textit{simulator} dan \textit{Microcontroller} untuk proses pengambilan data. Proses kerja dari sistem ini akan dijelaskan melalui diagram alur pada gambar \ref{fig: 3_2}.
	Selain itu, simulator ini memiliki 2 jenis data yang dapat diukur. Berikut penjelasan 2 macam jenis data tersebut.
\begin{figure}  [!htb]
	\captionsetup{justification=centering}
	\includegraphics[scale=0.25]{img/desainsistem.PNG}
	%\caption{Diagram alur kerja}
	\caption{Desain umum modul Simulator}
	\label{fig: 3_2}
\end{figure}
\vspace{1ex}

    \subsection{Data - Data Internal Simulator}
    \vspace{1ex}
    Jenis data yang pertama adalah data yang berasal dari dalam modul simulator, yaitu data - data seperti kecepatan mobil, informasi spasial seperti, sudut \textit{euler} \textit{(pitch,yaw,roll)} dan jarak relatif terhadap pinggir jalan, informasi tabrakan / \textit{colission}, serta informasi waktu respon / \textit{response time}. Data - data ini, disebut sebagai data internal, dikarenakan data - data tersebut bisa di ekstraksi langsung dari \textit{game engine}.
    
        \subsubsection{Kecepatan Mobil / \textit{Velocity}}
        
        Pada unity game engine, menggunakan library Unity. Bisa didapatkan secara langsung variabel - variabel yang berhubungan dengan kecepatan. Pada tugas akhir ini, dibutuhkan data kecepatan mobil relatif terhadap dunia atau biasa disebut \textit{global velocity}. Namun selanjutnya, data kecepatan global pun bisa dibagi menjadi 4 macam, yaitu : \textit{velocity} atau kecepatan kearah sumbu \textit{x}, \textit{velocity} atau kecepatan kearah sumbu \textit{y}, serta \textit{velocity} atau kecepatan kearah sumbu \textit{z}, ketiga hal tersebut bisa juga disebut kecepatan vektor \textit{x,y,z}. Dan yang terakhir, adalah \textit{velocity magnitude}, atau tingkat kebesaran suatu kecepatan berupa skalar. Contoh data ini bisa dilihat pada tabel pengujian \ref{tb:4_2}
        
        \subsubsection{Informasi Spasial}
        
        \begin{figure}  [!htb]
	        \captionsetup{justification=centering}
	        \includegraphics[scale=0.5]{img/relative-dist.jpg}
	        \caption{\textit{Relative Distance} \\ Oranye : Jarak dari \textit{Center of Mass} Mobil ke Batas Pinggir Kanan Jalan \\ Merah : Jarak dari \textit{Center of Mass} Mobil ke Batas Pinggir Kiri Jalan}
	        \label{fig: 3_19}
        \end{figure}
        
        Selain kecepatan, bisa didapatkan pula informasi spasial yang ada pada simulator. Pada tugas akhir ini, informasi spasial simulator adalah sudut euler yang merepresentasikan 6 derajat kebebasan atau biasa disebut \textit{6 Degree of Freedom (DoF))}, yaitu \textit{pitch, yaw,} dan \textit{roll} (gambar \ref{fig: 3_18}). Pada Tugas akhir ini, ketiga macam rotasi atau derajat kebebasan seluruhnya dimanfaatkan pada proses pembuatan desain jalan.
        
        \par Selanjutnya data spasial yang bisa didapatkan ialah posisi relatif kendaraan terhadap pinggir jalan (gambar \ref{fig: 3_19}). Dengan mengukur jarak terdekat dari titik pusat masa atau \textit{Center of Mass} mobil, ke batas pinggir kanan dan batas pinggir kiri jalan, dapat diketahui dimana posisi mobil di suatu lajur tersebut.
        
        \par Dengan menggabungkan dua macam jenis data tersebut, bisa didapatkan informasi yang sangat jelas tentang posisi dan orientasi dari kendaraan pengujian simulator ini. Yang nanti kedepannya sangat dibutuhkan untuk proses riset selanjutnya, yang akan memanfaatkan data - data tersebut.
        
        \subsubsection{Response Time}
        
        \begin{figure}  [!htb]
	        \captionsetup{justification=centering}
	        \includegraphics[scale=0.5]{img/jalur-tnp-mobil.JPG}
	        \caption{Response Time - IlustrasiJalur yang diharapkan - Kasus Tidak ada Mobil Lain}
	        \label{fig: 3_20}
        \end{figure}
        
        \begin{figure}  [!htb]
	        \captionsetup{justification=centering}
	        \includegraphics[scale=0.6]{img/jalur-dgn-mobil.JPG}
	        \caption{Response Time - Ilustrasi Jalur yang diharapkan - Kasus terdapat mobil lain}
	        \label{fig: 3_21}
        \end{figure}
        
        \textit{Response time} adalah metode pengukuran tingkat kewaspadaan pengemudi, dengan cara mendeteksi ketika pengemudi keluar dari lajur yang diharapkan. Lajur yang diharapkan disini, yang dimaksud ialah lajur sebelah kiri, dikarenakan desain simulator menggunakan model jalan yang ada di Indonesia. Berikut penjelasan tentang data \textit{Response time} pada gambar \ref{fig: 3_20} dan \ref{fig: 3_21}.
        
        Pada gambar \ref{fig: 3_20}, adalah gambar ilustrasi jalur yang diharapkan, pada kasus dimana tidak ada mobil lain didekat mobil pengujian. Warna hijau menandakan area yang di tandai oleh sistem simulator sebagai jalur yang diharapkan oleh sistem, artinya sistem mendeteksi bahwa mobil pada jalur yang telah sesuai serta pengemudi masih dalam keadaan waspada serta memiliki kontrol terhadap mobil. Apabila sistem mendeteksi mobil memasuki area yang berwarna merah, maka sistem akan melakukan pencatatan data yaitu kapan mobil mulai keluar dari jalur (tanggal dan waktu). Selanjutnya, sistem juga akan melacak mobil apabila mobil telah kembali ke area hijau, yang mana pada saat ini, sistem akan melakukan pencatatan data seberapa lama mobil telah keluar dari area hijau dengan \textit{unit} sekon, yang selanjutnya bisa didapatkan pula, seberapa lama mobil telah keluar dari area hijau dalam \textit{unit} hitungan \textit{frame}, dengan cara mengalikan waktu dalam sekon, dengan \textit{average framerate} dari simulator saat itu.
        
        \par Selanjutnya, pada gambar \ref{fig: 3_21}, adalah ilustrasi kasus dimana terdapat mobil didekat mobil uji, yang menyebabkan berubahnya jalur yang diharapkan. Apabila sistem mendeteksi terdapat mobil lain didekat mobil uji, sistem akan melakukan perubahan jalur yang diharapkan oleh sistem, metode penerapan pendeteksian seperti ini dapat diterapkan dengan banyak cara, salah satunya bisa dilakukan dengan membuat \textit{trigger box colission} disekitar mobil uji, yang mana ukurannya lebih besar dari \textit{bounding box} / ukuran mobil uji. Apabila terdapat suatu objek (seperti mobil lain), yang memasuki \textit{trigger box colission} dari mobil uji, maka bisa dilakukan perubahan pendeteksian jalur. Pada Tugas Akhir ini, digunakan metode ini untuk mendeteksi adanya mobil lain disekitar mobil uji, serta mendeteksi perlu terjadinya perubahan jalur yang diharapkan.
        
        \par Cara lain yang lebih mudah, namun tentunya dengan mengorbankan tingkat keakurasian pendeteksian yaitu adalah, dengan mengukur jarak mobil uji dengan mobil lain. Apabila jarak mobil uji dengan mobil lain ini dibawah nilai threshold, maka bisa dilakukan perubahan jalur yang diharapkan.
        
        \par Selanjutnya perlu digaris bawahi, bahwa sistem deteksi perubahan jalur ini juga sangat berkaitan dengan sistem deteksi terjadinya tabrakan. Sistem pendeteksian dan pencatatan data atau \textit{logging} dari \textit{colission event} ini akan dijelaskan pada subsubbab selanjutnya.
        
        \subsubsection{\textit{Colission Event}}
        
        \begin{figure}  [!htb]
	        \captionsetup{justification=centering}
	        \includegraphics[scale=0.65]{img/colliders.JPG}
	        \caption{2 Macam \textit{Colliders} Mobil - \textit{Box Collider} dan \textit{Mesh Collider}}
	        \label{fig: 3_22}
        \end{figure}
        
        \begin{figure}  [!htb]
	        \captionsetup{justification=centering}
	        \includegraphics[scale=0.7]{img/boundary-inside.JPG}
	        \caption{\textit{Boundary} dalam sirkuit}
	        \label{fig: 3_23}
        \end{figure}
        
        \begin{figure}  [!htb]
	        \captionsetup{justification=centering}
	        \includegraphics[scale=0.7]{img/boundary-outside.JPG}
	        \caption{\textit{Boundary} luar sirkuit}
	        \label{fig: 3_24}
        \end{figure}
        
         Pada subsubbab sebelumnya telah dijelaskan metode untuk mendeteksi keberadaan mobil lain di dekat mobil uji pada tugas akhir ini, yaitu dengan cara menggunakan \textit{trigger box event}. Pada subsubbab ini, \textit{colission event} yang dimaksud terdapat 2 macam. Yaitu yang pertama adalah, kejadian tabrakan dengan mobil lain, yang kedua kejadian tabrakan dengan pinggir jalan / \textit{boundary}. 
         
         \par Untuk proses pendeteksian tabrakan dengan mobil lain, dapat menggunakan \textit{mesh collider} yang sudah tersedia dengan mobil, untuk mempermudah proses deteksi, serta menyederhanakan \textit{physics interaction} antar mobil. Selain mobil uji, seluruh \textit{mesh collider} dari mobil lain ditentukan sebagai \textit{trigger}. Hal ini untuk menghindari terjadinya interaksi - interaksi yang tidak diinginkan ketika terjadi tabrakan antar mobil. 
         
         \par Dengan menggunakan fungsi \texttt{void onColissionEnter()} dan fungsi \texttt{void onTriggerEnter()} pada unity, dapat dilakukan pengecekan apabila terjadi overlap antar 2 \textit{GameObject} yang ada pada unity. Dengan menentukan \textit{GameObject} mana yang perlu dideteksi, sehingga langkah selanjutnya ialah mencatat data colission kedalam \textit{log file system}, 3 \textit{GameObject} yang perlu dideteksi adalah : \textit{Boundary} Luar / Batas Pinggir Kanan Jalan, \textit{Boundary} Dalam / Batas Pinggir Kiri Jalan, serta Mobil lain.
         
         \par Contoh data yang diambil pada \textit{colission event} bisa dilihat pada tabel \ref{tb:4_5}.
        
    
    \subsection{Data - Data Eksternal Simulator}
    \vspace{1ex}
    
     Selanjutnya Jenis data yang kedua adalah data yang berasal dari luar modul simulator, yaitu data - data seperti sinyal \textit{Electroencephalography (EEG)} / detak jantung pengemudi, dan atau data sinyal \textit{Electrooculography (EOG)} / data kedipan mata dari pengemudi, serta data citra wajah pengemudi menggunakan kamera. Data - data tersebut diatas, dikatakan sebagai data eksternal dikarenakan data - data tersebut didapatkan dari peralatan \textit{peripheral} yang dipasang pada modul simulator. Berhubung data EEG dan EOG merupakan data sinyal analog, maka diperlukannya suatu \textit{Analog to Digital Converter} atau ADC, supaya data bisa di rekam dalam \textit{log file}.
     Tentunya, perlu diperhatikan juga sampling rate dari ADC ini, sehingga bisa relevan dan dapat disesuaikan dengan data - data yang lain.
     
        \subsubsection{\textit{Serial Communication} Melalui \textit{port} COM}
        Pada tugas akhir ini, \textit{serial communication} melalui \textit{port} COM, dilakukan menggunakan \textit{Python}. Diagram \textit{flow chart} dari proses komunikasi data serial arduino ke PC, bisa di lihat pada gambar \ref{fig: 3_25}.
        \par Proses dari modul pembacaan data serial dimulai dengan pembuatan \textit{file} dengan nama tanggal dijalankannya modul tersebut. Hal ini dilakukan untuk mengetahui kapan waktu dan tanggal modul pembacaan dijalankan. Selain itu, hal ini memastikan bahwasanya file - file dapat dipisahkan atau disortir berdasarkan waktu dan tanggal, yang artinya data - data yang dicatat, tidak akan bercampur aduk dengan data - data percobaan yang sebelumnya.
        \par Dengan mengasumsikan bahwa arduino akan melakukan pengiriman data, \textit{python} akan terus membaca data dari arduino, dengan maksimum \textit{baudrate} yang telah ditentukan. Kemudian, setelah proses pembacaan dilakukan, akan dilakukan proses \textit{decode}. Proses \textit{Decode} ialah proses dimana \textit{python} akan melakukan konversi data biner, menjadi data - data diskrit \textit{integer} / bilangan bulat, yang kemudian hasil dari proses \textit{decode}, akan dilakukan proses append pada \textit{file} yang telah dibuat sebelumnya. 
        \par Informasi lebih jelas tentang proses \textit{append file} dan \textit{log file system} akan dijelaskan lebih lanjut pada sub-bab \ref{logfilesystem}
        
        \subsubsection{Citra Wajah Pengemudi / \textit{Webcam}}
        Pada \textit{Unity Game Engine}, sistem mensupport pembacaan informasi kamera webcam sebagai salah satu bagian dari \textit{library mobile camera information}, yang artinya adalah, apabila platform dari simulator menggunakan PC dengan OS \textit{Windows}, dan terdapat \textit{peripheral Webcam} yang terpasang, Unity dapat mendeteksi \textit{webcam} tersebut sebagai \textit{Imaging Device}.
        \par Sebagai \textit{Imaging Device}, data yang diperoleh unity dari \textit{webcam} ditangkap sebagai \textit{texture information}, atau informasi texture dari suatu \textit{GameObject}.
        
        \begin{figure}  [!htb]
	        \captionsetup{justification=centering}
	        \includegraphics[scale=0.62]{img/webcam-texture.JPG}
	        \caption{Kubus yang memiliki \textit{texture information} dari \textit{Webcam / Imaging Device}}
	        \label{fig: 3_26}
        \end{figure}
        
       Dikarenakan informasi yang ditangkap oleh unity merupakan suatu \textit{texture}, maka \textit{texture} tersebut bisa di pasangkan ke suatu \textit{GameObject}, agar memiliki tampilan dari \textit{webcam}.  Bisa dilihat, pada gambar \ref{fig: 3_26}, ketika informasi webcam di pasangkan dengan kubus.
       \par Namun pada tugas akhir ini, yang dibutuhkan bukanlah informasi \textit{texture} yang bisa digunakan oleh game ini. Yang dibutuhkan adalah \textit{frame - frame} / gambar wajah dengan \textit{timestamp} yang sesuai. Maka langkah selanjutnya adalah mengekstrak informasi \textit{texture} tersebut keluar dari unity. Hal ini mudah dilakukan dengan melakukan fungsi \textit{encoding} ke tipe \textit{file} yang berekstensi \textit{PNG}. Setelah encoding dilakukan, \textit{file stream} akan menuliskan data hasil \textit{encoding} menjadi suatu gambar lengkap dengan \textit{timestamp} nya.
       \par Hasil pengujian pengambilan data citra wajah pengemudi menjadi frame - frame yang telah di \textit{encode} menjadi PNG bisa di lihat pada gambar \ref{fig:4.2}
    
\section{Desain Lajur Simulator}
\vspace{1ex}

    \par Pada tugas akhir ini, terdapat 4 macam lajur yang dapat dipilih oleh pengemudi. 4 Macam lajur tersebut berfungsi sebagai basis riset untuk mengeliminasi terjadinya bias atau pengaruh yang dihasilkan oleh perbedaan jumlah lajur yang digunakan.
    
\begin{figure}  [!htb]
	\captionsetup{justification=centering}
	\includegraphics[scale=0.31]{img/2lj_short_oh.jpg}
	\caption{Desain sirkuit 2 lajur dengan jarak yang pendek dan kompleksitas yang rendah}
	\label{fig: 3_3}
\end{figure}
\vspace{1ex}

\begin{figure}  [!htb]
	\captionsetup{justification=centering}
	\includegraphics[scale=0.53]{img/2lj_long_oh.jpg}
	\caption{Desain sirkuit 2 lajur dengan jarak yang panjang dan kompleksitas yang tinggi}
	\label{fig: 3_4}
\end{figure}
\vspace{1ex}

\begin{figure}  [!htb]
	\captionsetup{justification=centering}
	\includegraphics[scale=0.53]{img/3lj_oh.jpg}
	\caption{Desain sirkuit 3 lajur}
	\label{fig: 3_5}
\end{figure}
\vspace{1ex}

\begin{figure}  [!htb]
	\captionsetup{justification=centering}
	\includegraphics[scale=0.53]{img/4lj_oh.jpg}
	\caption{Desain sirkuit 4 lajur}
	\label{fig: 3_6}
\end{figure}
\vspace{1ex}

    \par Berikut pada Gambar \ref{fig: 3_3}, \ref{fig: 3_4}, \ref{fig: 3_5}, dan \ref{fig: 3_6}, adalah tampak atas dari desain - desain lajur yang digunakan pada simulator ini, seluruh lajur didesain dengan bentuk \textit{closed-loop circuit}. Hal ini bertujuan agar mengurangi ukuran aset jalan yang akan digunakan. Selain itu desain seperti ini dapat berdampak pada durasi pengambilan data, desain sirkuit yang \textit{closed-loop} memungkinkan kegiatan proses pengambilan data supaya tidak bergantung pada panjang jalan yang digunakan.
    \par Tabel \ref{tb:3_1} adalah detil informasi dari desain jalan / sirkuit yang dibuat pada simulator ini. 
    
\begin{table}[]
\caption{Tabel Informasi Sirkuit Jalan}
\label{tb:3_1}
\begin{tabular}{|c|c|c|c|c|c|}
\hline
\multirow{2}{*}{No.} & \multirow{2}{*}{\begin{tabular}[c]{@{}c@{}}Jumlah Lajur\\ (Unit)\end{tabular}} & \multirow{2}{*}{\begin{tabular}[c]{@{}c@{}}Jarak\\ (Unit)\end{tabular}} & \multirow{2}{*}{\begin{tabular}[c]{@{}c@{}}Lebar\\ (Unit)\end{tabular}} & \multicolumn{2}{c|}{\begin{tabular}[c]{@{}c@{}}Kompleksitas\\ (Tampak Atas)\end{tabular}} \\ \cline{5-6} 
                     &                                                                                &                                                                         &                                                                               & \textit{CW Curve}                           & \textit{CCW Curve}                          \\ \hline
1                    & 2                                                                              & 43,145                                                                  & 12                                                                            & 4                                           & 0                                           \\ \hline
2                    & 2                                                                              & 260.226                                                                 & 12                                                                            & 6                                           & 10                                          \\ \hline
3                    & 3                                                                              & \multicolumn{1}{l|}{260.226}                                            & 16                                                                            & 6                                           & 10                                          \\ \hline
4                    & 4                                                                              & 411.346                                                                 & 20                                                                            & 11                                          & 12                                          \\ \hline
\end{tabular}
\end{table}
    
\section{Desain \textit{Behaviour} dari Kendaraan Lain}
\vspace{1ex}

    \par Pada Tugas akhir ini, Desain dari \textit{AI Behavior} kendaraan lain cukup mengikuti lajur sirkuit sesuai dengan jalur yang telah didefinisikan oleh \textit{Bézier curve}, kemudian pada script \textit{follower} yang ada di unity, dikarenakan desain lajur yang \textit{closed-loop}, perlu di definisikan arah dari kendaraan tersebut, dilihat dari tampak atas, apakah kendaraan memiliki \textit{clockwise path} atau \textit{counter-clockwise path}.
    \par Selain arah dari kendaraan, diperlukannya pendefinisian kecepatan kendaraan tersebut, pada Tugas Akhir ini kecepatan kendaraan ditentukan dengan randomisasi dalam suatu rentang tiap kali simulator dijalankan.
    \par Informasi detil dari desain AI kendaraan lain di \textit{unity}, dapat dilihat pada tabel \ref{tb:3_2}

\begin{table}[]
\caption{Tabel Informasi AI Tiap - Tiap Tipe Lajur}
\label{tb:3_2}
\begin{tabular}{|c|c|c|c|c|c|c|}
\hline
\multirow{2}{*}{No.} & \multirow{2}{*}{Lajur} & \multirow{2}{*}{\begin{tabular}[c]{@{}c@{}}Jumlah\\ Mobil\end{tabular}} & \multicolumn{2}{c|}{\textit{AI Rotation}}             & \multicolumn{2}{c|}{\textit{\begin{tabular}[c]{@{}c@{}}Velocity\\ (Unit/Frame)\end{tabular}}} \\ \cline{4-7} 
                     &                        &                                                                         & \textit{CW}               & \textit{CCW}              & \textit{Min}                                  & \textit{Max}                                  \\ \hline
1                    & \textit{2 (Short)}     & 1                                                                       & \xmark     & \checkmark & 5.0f                                          & 15.0f                                         \\ \hline
2                    & \textit{2 (Long)}      & 2                                                                       & \checkmark & \checkmark & 0.1f                                          & 0.9f                                          \\ \hline
3                    & 3                      & 3                                                                       & \checkmark & \checkmark & 0.1f                                          & 0.9f                                          \\ \hline
4                    & 4                      & 4                                                                       & \checkmark & \checkmark & 0.01f                                         & 0.1f                                          \\ \hline
\end{tabular}
\end{table}

\section{Alur Kerja}
\vspace{1ex}

Pembuatan tugas akhir ini dibagi menjadi beberapa tahapan, yaitu:
    \begin{enumerate}[nolistsep]
	
	\item \nohyphens{Pembuatan Simulasi Menggunakan \textit{Unity Game Engine}}
	\item Pengaturan dan Konfigurasi \textit{Steering Wheel Controller} 
	\item Pembuatan Modul Pengambilan Data dengan \textit{Microcontroller}
	\item Penggabungan Seluruh Sistem Menjadi Satu Modul
	
	\end{enumerate}

\section{Pembuatan Simulasi Menggunakan \textit{Unity Game Engine}}
\vspace{1ex}
   Pada Tugas Akhir ini, proses pembuatan simulasi menggunakan \textit{Unity Game Engine}. Namun tidak hanya menggunakan \textit{Unity Game Engine} saja, tentunya selain memanfaatkan \textit{unity editor}, juga diperlukan \textit{source code editor}, pada tugas akhir ini menggunakan \textit{Visual Studio 2017}. Selain itu terdapat \textit{tools - tools} lain di luar \textit{unity} yang dapat mempermudah proses pembuatan Simulator ini.
	
	\subsection{Pembuatan Sirkuit Jalan}
	\vspace{1ex}
	
	\begin{figure} [!htb]
	    \captionsetup{justification=centering}
	    \includegraphics[scale=0.25]{img/shanghai-ic.png}
	    \caption{\textit{Shanghai International Circuit\cite{cit:14}}}
	    \label{fig: 3_10}
    \end{figure}
	
	\begin{figure} [!htb]
	    \captionsetup{justification=centering}
	    \includegraphics[scale=0.36]{img/bezier-curve-jalan.JPG}
	    \caption{5 Macam \textit{Bézier curve} sirkuit (dari kiri ke kanan) : batas kiri jalan, \textit{Vehicle AI Path - Counterclock Wise}, tengah lajur / marka jalan, \textit{Vehicle AI Path - Clock Wise}, serta batas kanan jalan }
	    \label{fig: 3_9}
    \end{figure}
	
	Proses pembuatan sirkuit jalan adalah proses yang paling memakan waktu pada tugas akhir ini. Yang pertama perlu dilakukan adalah mendefinisikan \textit{path} jalan menggunakan \textit{Bézier curve}, atau jalur yang akan digunakan sebagai \textit{base} dari \textit{3d mesh} jalan atau lajur itu sendiri, salah satu contohnya adalah pada gambar \ref{fig: 3_9}, bisa dilihat terdapat 5 macam kurva bezier yang dapat ditemukan. Semua 5 kurva tersebut dibutuhkan untuk proses kalkulasi informasi jarak serta informasi \textit{colission} yang nantinya akan disimpan kedalam \textit{log file}.
	\par Menggunakan \textit{Shanghai International Circuit} dari \textit{Formula 1}, sebagai referensi bentuk dari sirkuit, diperlukannya penyesuaian tingkat kelengkungan dari belokan yang ada pada sirkuit referensi. Untuk menghindari \textit{mesh overlap}, maka tiap - tiap tikungan yang tajam, perlu diperhalus sedemikian hingga \textit{mesh overlap} agar tidak terjadi. Namun tidak terlalu dikurangi sehingga pengemudi tetap waspada terhadap tikungan tersebut, serta supaya pengemudi mengurangi kecepatan sebelum melakukan belokan tersebut.
	\par Hal ini penting, dikarenakan pada saat proses pengujian berjalan, mobil tidak diharapkan untuk keluar dari lajur pengujian.
	    
	    \subsubsection{\textit{Editing Tools : Path Creator}}
	    \vspace{1ex}
	    
	    \begin{figure} [!htb]
	        \captionsetup{justification=centering}
	        \includegraphics[scale=0.47]{img/bezier-curve-cp.JPG}
	        \caption{\textit{Bézier curve} tampak atas, merah : \textit{anchor points}, biru : \textit{control points}}
	        \label{fig: 3_11}
        \end{figure}
        
        \begin{figure} [!htb]
	        \captionsetup{justification=centering}
	        \includegraphics[scale=0.35]{img/bezier-curve-cp2.JPG}
	        \caption{\textit{Bézier curve} pada \textit{3d plane}}
	        \label{fig: 3_12}
        \end{figure}
	    
	    Untuk memudahkan proses pembuatan \textit{Bézier curve}, pada unity. Penulis menggunakan \textit{tools} yang tersedia di unity yang berfungsi untuk mempermudah kalkulasi matematis serta proses \textit{scripting} yang ada pada unity untuk menghasilkan kurva tersebut. Tools tersebut ialah \textit{Path Creator} oleh Sebastian Lague, Path Creator ini dipilih dikarenakan, memudahkan dalam proses manipulasi kurva \textit{Bézier}. Dikarenakan pada \textit{unity} tidak mensupport kurva \textit{Bézier} secara natif, maka biasanya para \textit{game developer}, membuat \textit{script} mereka masing - masing untuk membuat kurva \textit{Bézier}, dengan melakukan kalkulasi persamaan matematis pada script mereka tersebut, kemudian diperlukannya manipulasi titik - titik kontrol sesuai dengan \textit{script} tersebut.
	    \par Namun \textit{Path Creator} memungkinkan penulis untuk melewati proses kalkulasi matematis serta pendefinisian titik - titik kontrol, langsung ke proses manipulasi \textit{drag and drop} titik - titik kontrol, pada \textit{Unity Editor}.
	    \par Pada gambar \ref{fig: 3_11} dan \ref{fig: 3_12}, bisa dilihat titik - titik yang dapat dimanipulasi menggunakan \textit{Path Creator}. Warna merah menunjukkan \textit{anchor points}, yang artinya titik tersebut merupakan titik - titik statis pada suatu persamaan kurva, dengan memanipulasi lokasi dari \textit{anchor points} pada 3 dimensi, dapat didapatkan bentuk \textit{approximation} dari target sirkuit diinginkan. Tentunya, semakin banyak \textit{anchor points} yang digunakan, semakin akurat pula \textit{approximation} jalan yang dibuat. Namun tentunya, kurva jalan menjadi lebih kompleks serta lebih sulit untuk dimanipulasi.
	    \par Kemudian, warna biru menunjukkan \textit{control points}, \textit{control points} ini menentukan lengkungan yang diharapkan diantara tiap - tiap \textit{anchor points}, tentunya agar didapatkankannya lengkungan yang sesuai dengan belokan yang ada pada sirkuit referensi, titik - titik inilah yang akan dapat dimanipulasi.
	    
	    \subsubsection{\textit{Plane Projection} serta \textit{Normal Vector}}
        
        Pada Tugas Akhir ini, informasi tentang \textit{plane projection} yang ada pada kurva \textit{Bézier} sangat penting. Hal ini dapat menunjukkan informasi tentang kekasaran medan atau \textit{level} ketinggian dari suatu daerah pada \textit{terrain} yang ada pada jalan.
        
        \par Pada Tugas akhir ini, jalan yang memiliki 2 dan 3 lajur, memiliki \textit{plane projection} terhadap \textit{XZ Plane}, artinya medan yang ada pada kedua macam jumlah lajur tersebut memiliki kedataran yang seragam, tidak memiliki tanjakan ataupun turunan. 
        
        \par Sedangkan, pada jalan yang memiliki 4 lajur, dengan memanfaatkan semua sumbu \textit{XYZ} maka dapat dihasilkan jalur yang memiliki tanjakan dan turunan.
        
        \begin{figure} [!htb]
	        \captionsetup{justification=centering}
	        \includegraphics[scale=0.35]{img/pp1.JPG}
	        \caption{\textit{Normal Force} Jalan (\textit{Yaw} 0 Derajat)}
	        \label{fig: 3_13}
        \end{figure}
        
        \begin{figure} [!htb]
	        \captionsetup{justification=centering}
	        \includegraphics[scale=0.35]{img/pp2.JPG}
	        \caption{\textit{Normal Force} Jalan (\textit{Yaw} 60 Derajat)}
	        \label{fig: 3_14}
        \end{figure}
        
        Kemudian selanjutnya adalah, penentuan \textit{normal force}. Pada \textit{path creator}, parameter yang dapat dikontrol adalah \textit{global angle}, dari normal force tersebut. Hal ini menentukan sudut rotasi / \textit{yaw}, dari jalan yang dibuat pada path yang ditentukan.
        \par Berikut pada gambar \ref{fig: 3_13} dan gambar \ref{fig: 3_14}, perbandingan efek dari \textit{normal force} yang ada di jalan. Hal ini penting diperhatikan untuk proses desain serta pembuatan jalan, yang memanfaatkan seluruh 6 derajat kebebasan \textit{(Degree of Freedom)}, \textit{pitch, yaw,} dan \textit{roll}
	
	\subsection{Pembuatan \textit{Terrain}}
	\vspace{1ex}
	
	Medan atau \textit{terrain}, merupakan kanvas kosong dari sebuah \textit{scene}, yang dapat diisi dengan berbagai macam objek - objek estetis untuk meningkatkan realisme dan imersivitas pada proses simulasi. Pada tugas akhir ini medan yang dibuat pada tiap lajur tidak terlalu berpengaruh terhadap proses pengambilan data itu tersendiri, namun merupakan salah satu \textit{design choice} atau proses pemilihan desain agar meningkatkan kualitas dari \textit{user experience}.
	
	\par Setelah proses pembuatan jalan selesai, untuk meningkatkan estetika, diperlukannya \textit{terrain} yang mengelilingi jalan. Pada \textit{Unity,} objek - objek \textit{non-colission} dapat ditaruh untuk meningkatkan estetika tersebut. Seperti contohnya, pohon - pohon dan semak - semak atau rerumputan. Selain itu, dapat diterapkan perubahan kontur dari \textit{terrain}. Seperti contohnya dapat diterapkan proses peninggian sebagian dari \textit{terrain} untuk menghasilkan bukit, atau bahkan sebaliknya, proses penurunan dari sebagian \textit{terrain} dapat menghasilkan lembah.
	
    Selain itu, pada \textit{unity} dapat ditambahkannya tekstur pada \textit{terrain}. Tekstur disini yang dimaksud ialah, warna atau gambar yang dapat diterapkan pada \textit{terrain} agar menghasilkan corak pada \textit{terrain}. Seperti contohnya tekstur dari tanah coklat, atau tekstur warna pasir kuning, dan sebagainya.
    
    \begin{figure} [!htb]
	    \captionsetup{justification=centering}
	    \includegraphics[scale=0.35]{img/terrain.JPG}
	    \caption{\textit{Terrain} setelah ditambahkan bukit, dan pepohonan}
	    \label{fig: 3_17}
    \end{figure}
    
    Selanjutnya, perlu di garis bawahi juga, pada Tugas Akhir ini, salah satu \textit{design choice} merupakan jalan berbentuk sirkuit, maka dari itu terrain, untuk mengakomodasi hal tersebut di terapkan bukit yang mengelilingi sirkuit jalan, untuk memberikan kesan bahwasanya mobil tidak dapat keluar dari lajur.
    
    \subsection{\textit{Menu} dan \textit{User Interface}}
    
    Proses pembuatan Menu dibutuhkan untuk memudahkan pemilihan jumlah lajur nanti setelah simulator menjalani proses \textit{build}. Menu disini dibutuhkan agar tidak terlalu kompleks, namun cukup agar menu dapat menjelaskan serta merepresentasikan tampilan atau tombol yang ditekan. Pemilihan yang intuitif serta fungsionalitas yang berjalan sudah cukup untuk proses ketika di menu.
    \par Perlu diperhatikan juga, dapat diterapkan pula \textit{asynchronous loading}, pada menu ini. Artinya, \textit{scene} atau jenis lajur yang dipilih, akan di muat sebelum tombol ditekan. Supaya proses \textit{loading} akan menjadi lebih cepat.
    
    \begin{figure} [!htb]
	    \captionsetup{justification=centering}
	    \includegraphics[scale=0.35]{img/menuUI.JPG}
	    \caption{\textit{User Interface} dari Menu}
	    \label{fig: 3_15}
    \end{figure}
    
    Gambar \ref{fig: 3_16} adalah, gambar yang menunjukkan hasil menu yang dibuat. Terdapat 4 tombol yang bisa ditekan, tiap - tiap tombol memiliki penjelasan \textit{text} yang jelas, yaitu merepresentasikan tombol yang akan membawa \textit{user} menuju \textit{scene} atau lajur yang sesuai dengan text tersebut, seperti 2 lajur pendek, 2 lajur panjang, 3 lajur, serta 4 lajur. Selain itu, dipasang pula \textit{background canvas} serta logo ITS pada \textit{User Interface} Menu.
    
    \subsection{\textit{Log File System} dan \textit{Kalkulasi Data}} 
    \label{logfilesystem}
    
    \textit{Log File System} adalah salah satu komponen penting dari Tugas Akhir ini, agar didapatkan suatu \textit{Log File System}, beberapa macam komponen dibutuhkan dari sistem ini (gambar \ref{fig: 3_15}).
    
    \begin{figure} [!htb]
	    \captionsetup{justification=centering}
	    \includegraphics[scale=0.5]{img/logfile.JPG}
	    \caption{Diagram \textit{Log File System}, proses ekstraksi data, dan proses \textit{append data}}
	    \label{fig: 3_31}
    \end{figure}
    
    Yang pertama adalah, proses ekstraksi data. Proses ini sangat mudah dikarenakan informasi - informasi yang dibutuhkan sudah tersedia dan tertulis pada variabel - variabel yang dipakai pada simulasi serta komunikasi \textit{serial data}. 
    \par Setelah proses ekstraksi data yang dibutuhkan, selanjutnya dibuatlah komunikasi \textit{file stream} dari \textit{unity} ke \textit{OS file system}, pada Tugas Akhir ini, OS yang digunakan adalah Windows 8.1. Komunikasi file stream yang dibuat bertugas untuk mengecek apakah \textit{path} dari direktori yang dituju sudah ada, apabila belum tersedia folder yang diharapkan, maka \textit{file stream}, akan membuat folder pada direktori yang diinginkan dan membuat \textit{file - file} sesuai dengan \textit{output data} yang ada.
    \par Langkah selanjutnya adalah \textit{file stream} akan membuka \textit{file} yang telah dibuat tersebut, kemudian melakukan \textit{append}, atau penambahan data kedalam \textit{file} tersebut. Jenis \textit{Append} yang pertama kali dilakukan adalah melakukan \textit{append header}, atau informasi - informasi pada ujung tabel (label), serta informasi \textit{separator} dari kolom. Pada Tugas Akhir ini, seluruh \textit{output file}, menggunakan \textit{separator} kolom berupa \textit{tab} atau simbol \textit{\textbackslash t} sedangkan separator baris berupa \textit{line break} atau simbol \textit{\textbackslash n}
    

\section{Pengaturan dan Konfigurasi \textit{Steering Wheel Controller}}
\label{steeringwheelconf}
\vspace{1ex}
    \begin{figure}  [!htb]
	    \captionsetup{justification=centering}
	    \includegraphics[scale=0.51]{img/steeringwheel.jpg}
	    %\caption{Diagram alur kerja}
	    \caption{Diagram Steering Wheel yang digunakan, \textit{Sumber : Buku Manual PXN Steering Wheel}}
	    \label{fig: 3_28}
    \end{figure}

	Konfigurasi kontroller yang perlu dilakukan adalah salah satunya melakukkan \textit{mapping} tombol - tombol dari keyboard ke tombol - tombol serta perangkat analog dari \textit{steering wheel controller} yang digunakan, seperti contohnya perangkat analog yang perlu dilakukan \textit{sampling} adalah pedal, \textit{sampling} yang dimaksud adalah melakukan pembagian data voltase analog yang ada di pedal menjadi nilai - nilai diskrit yang kemudian dapat dikonversikan ke kecepatan mobil, sesuai nilai - nilai diskrit yang didapatkan. Selain itu juga perlu dilakukan kalibrasi mode sudut dari steering wheel yang memiliki 2 macam mode, yaitu mode 270 derajat putaran, serta 900 derajat putaran. 
	Pada Tugas akhir ini, \textit{steering wheel} akan menggunakan mode 270 derajat, serta proses pergantian gigi \textit{(gear shift)} menggunakan proses manual tanpa kopling. Hal ini dikarenakan keterbatasan perangkat keras yang digunakan.
	
	\begin{figure}  [!htb]
	    \captionsetup{justification=centering}
	    \includegraphics[scale=0.3]{img/driverinput.JPG}
	    %\caption{Diagram alur kerja}
	    \caption{Grafik hasil \textit{plotting} data \textit{input} pengemudi}
	    \label{fig: 3_29}
    \end{figure}
    
    Berikut pada gambar \ref{fig: 3_28} adalah \textit{key map} dari sistem simulator tugas akhir ini. proses kalibrasi pedal gas dan rem disambungkan ke \textit{vertical axis / axis y} dari \textit{unity controller}, maka bisa didapatkan nilai dari pedal atau rem tersebut, berupa data \textit{float} yang memiliki rentang 0 hingga 1. Sedangkan kalibrasi \textit{steering wheel} disambungkan dengan \textit{horizontal axis / axis x} dari \textit{unity controller}, yang memiliki data berupa \textit{float} dengan rentang -1 hingga 1.
    \par Kemudian, data \textit{float} tersebut, dapat dilakukan proses kalkulasi untuk mendapatkan sudut dari pedal gas dan rem, serta sudut dari \textit{steering wheel}, sehingga dapat dilakukan proses \textit{plotting} seperti berikut pada gambar \ref{fig: 3_29}
    
    Selain \textit{vertical axis} dan \textit{horizontal axis}, terdapat pula tombol - tombol lain dari \textit{steering wheel controller} yang digunakan. Yaitu tombol \textbf{X} untuk menyalakan mesin dari mobil \textit{(starter)}, selain itu ada tombol \textit{R-Paddle} dan \textit{L-Paddle} untuk menaikkan dan menurunkan gigi mesin secara berurutan. Proses ini, tidak memerlukan kopling, cukup menekan tombol \textit{L-R-Paddle} untuk menaikkan atau menurunkan gigi mobil.
    
    \par Di tengah \textit{steering wheel} juga terdapat \textit{switch} untuk mengganti mode \textit{steering} yaitu mode 270 derajat atau 900 derajat.
\section{Pembuatan Modul Pengambilan Data dengan \textit{Microcontroller}}
\label{microcontroller}
\vspace{1ex}

    \begin{figure}  [!htb]
	    \captionsetup{justification=centering}
	    \includegraphics[scale=0.2]{img/sinyaldangambar.JPG}
	    %\caption{Diagram alur kerja}
	    \caption{Diagram Sinkronisasi \textit{capture rate} kamera dengan \textit{sampling rate} dari sensor arduino}
	    \label{fig: 3_30}
    \end{figure}
    
    Pembuatan modul pengambilan data serial seperti EEG dan ECG, dapat dilakukan menggunakan Arduino, sehingga perlu disiapkan port USB. Sedangkan pengambilan data berupa gambar wajah pengemudi menggunakan kamera, dapat disambungkan langsung ke \textit{Unity Game Engine} dan dapat langsung disimpan ke \textit{harddrive} komputer simulator. Hal ini dilakukan agar data yang masuk berupa gambar wajah pengemudi serta sinyal - sinyal yang di peroleh oleh \textit{microcontroller} dapat di lakukan pengolahan dan proses analisa.
    Perlu di perhatikan juga \textit{sampling rate} dari sistem pengambilan sinyal arduino, dengan \textit{capture rate} dari \textit{webcam}, kedua data tersebut perlu di sinkronkan dengan memperhatikan kedua nilai tersebut. \textit{Sampling rate} dari arduino disini bisa didapatkan dari \textit{datasheet} sensor yang digunakan, kemudian dilakukan kalkulasi dengan \textit{margin of error} yang disebabkan oleh kegagalan transmisi data dari arduino ke komputer simulator melalui kabel USB, agar didapatkan hasil yang dapat diterima. Sedangkan \textit{capture rate} dari kamera merupakan seberapa banyak gambar atau \textit{frame} yang diambil oleh kamera tiap \textit{frame} yang di proses oleh simulator. Pada Tugas Akhir ini, target capture rate adalah 1:1. Yang artinya, kamera akan mengambil gambar wajah pengemudi tiap frame game tersebut dijalankan. Berikut pada gambar \ref{fig: 3_30} adalah diagram dari penjelasan sinkronisasi kedua data tersebut, menjadi suatu modul analisis sinyal dengan gambar wajah.
    
\section{Penggabungan Seluruh Sistem Menjadi Satu Modul}
\vspace{1ex}
    Langkah terakhir adalah mengintegrasikan seluruh sistem menjadi satu modul yang utuh, supaya dapat melakukan pengolahan data dengan irisan waktu \textit{(t)} yang bersamaan. Keluaran yang diharapkan adalah, simulator dapat melakukan pengambilan data analog, data citra gambar, serta data - data internal simulasi secara bersamaan dan integrasi. Seperti yang dijelaskan pada bab \ref{logfilesystem} hingga bab \ref{steeringwheelconf}, Penggabungan Sub - sub modul, menjadi satu modul utuh adalah seperti berikut
    
    \begin{enumerate}[nolistsep]
	
	\item Sub-modul pengambilan data internal, mencakup :
	    \begin{enumerate}[nolistsep]
	        \item Sub-Modul Kecepatan Mobil
	            \begin{enumerate}
	                \item Kecepatan Vektor Sumbu \textit{x,y,z}
	                \item \textit{Magnitude} Kecepatan Vektor
	            \end{enumerate}
	        \item Sub-Modul Informasi Spasial
	            \begin{enumerate}
	                \item Posisi Relatif Mobil terhadap jalan
	                \item Sudut Euler dan \textit{6 Degree of Freedom} - Pitch, Yaw, Roll
	            \end{enumerate}
	        \item Sub-Modul Respon Pengemudi
	            \begin{enumerate}
	                \item Informasi Respon Pengemudi Ketika kendaraan keluar jalur
	                \item Informasi \textit{Input} dari Steering Wheel Controller (Sudut \textit{Steering Wheel} dan Nilai Tekanan Pedal Gas dan Rem)
	            \end{enumerate}
	    \end{enumerate}
	\item Sub-modul pengambilan data eksternal, mencakup :
	    \begin{enumerate}[nolistsep]
	        \item Sub-Modul Sinyal dan Citra Wajah
	            \begin{enumerate}
	                \item Sinyal dari sensor, bisa berupa ECG, EEG, atau EOG
	                \item Citra Wajah tiap frame dari kamera
	            \end{enumerate}
	    \end{enumerate}
	
	\end{enumerate}
    
Pada tugas akhir ini, sub - sub modul dibentuk berdasarkan kebutuhan proses pengujian, namun untuk riset kedepannya, proses pembuatan serta sinkronisasi data - data sub-modul bisa dapat dilakukan penyesuaian sesuai dengan kebutuhan riset tersebut yang menggunakan data dari simulator ini.
    
    
    
% GAMBAR KALIBRASI STEERING WHEEL
%\begin{figure} [!htb]
%	\captionsetup{justification=centering}
%	\includegraphics[scale=0.5]{img/steering-wheel-calibration.jpg}
%	\caption{Informasi Steering Wheel }
%	\label{fig: 3_7}
%\end{figure}

%GAMBAR FLOW CHART KOMUNIKASI PYTHON
\begin{figure}  [!htb]
	        \captionsetup{justification=centering}
	        \includegraphics[scale=0.7]{img/serial-comm.jpg}
	        \caption{Diagram \textit{Flow Chart} Komunikasi Serial Melalui \textit{port} COM}
	        \label{fig: 3_25}
\end{figure}
\cleardoublepage
\chapter{PENGUJIAN DAN ANALISA}
\vspace{1ex}

\section*{}
Pada bab ini dipaparkan hasil pengujian dari Tugas Akhir serta analisa dari desaim sistem simulator dan implementasinya. Pengujian dibagi menjadi lima bagian antara lain:
\vspace{1ex}
\begin{enumerate}[nolistsep]
	\item Pengujian \textit{User Interface} - Pemilihan Jumlah Lajur
	\item Pengujian Pengambilan Data - Kecepatan
	\item Pengujian Pengambilan Data - Informasi Spasial
	\item Pengujian Pengambilan Data - \textit{Response Time} dan \textit{Input} Pengemudi
	\item Pengujian Pengambilan Data - Citra \textit{Webcam}
	\item Pengujian Pengambilan Data - \textit{Serial Data USB}
	\item Pengujian Respon Sinyal dari \textit{Steering Wheel Controller} terhadap simulator
	\item Pengujian \textit{User Experience} / UX Pengguna
	
	\vspace{1ex}

\end{enumerate}
Dengan dilaksanakannya beberapa pengujian tersebut, sehingga dapat ditarik kesimpulan dari pelaksanaan tugas akhir ini.
\vspace{1ex}

\section{Pengujian \textit{User Interface} - Pemilihan Jumlah Lajur}
\vspace{1ex}

Pada pengujian \textit{user interface}, \textit{user} dapat memilih jumlah lajur yang akan digunakan. Terdapat 4 \textit{scene} yang dapat dipilih dari \textit{user interface}, yaitu 2 lajur (pendek), 2 lajur (panjang) , 3 lajur, serta 4 lajur.

Kesimpulan pada pengujian \textit{user interface} (Gambar \ref{fig:4.1}), tombol menu lajur yang ditampilkan dan yang dipilih oleh \textit{user} sudah berkorelasi dengan \textit{scene} yang dimuat oleh game.

\section{Pengujian Pengambilan Data - Kecepatan}
\vspace{1ex}

Pada pengujian pengambilan data kecepatan, didapatkan data seperti pada tabel \ref{tb:4_2}, data yang didapatkan merupakan data mentah \textit{(raw data)} tanpa pengolahan atau kalkulasi tambahan, dari \textit{Unity Game Engine} dengan cara mengakses komponen \textit{RigidBody Game Object} mobil, dengan fungsi \texttt{GetComponent<Rigidbody>()}, sehingga didapatkan \textit{object variable} yaitu \texttt{velocity} untuk mendapatkan kecepatan vektor \textit{x,y,z}, serta \texttt{velocity.magnitude} untuk mendapatkan besaran dari kecepatan vektor tersebut. Pengambilan / \textit{sampling} data dilakukan tiap frame, sehingga tabel \ref{tb:4_2} hanya menampilkan data kurang lebih $\frac{1}{2}$ detik.

Dapat disimpulkan pengujian ini (tabel \ref{tb:4_2}) dapat menghasilkan data dengan tingkat akurasi yang tinggi, dikarenakan data dapat diambil dalam rentang waktu yang cukup kecil tiap framenya (Unit tiap Frame).

\section{Pengujian Pengambilan Data - Informasi Spasial}
\vspace{1ex}

Informasi spasial pada pengujian ini terdapat 2 macam data, yaitu posisi relatif kendaraan terhadap jalan, serta sudut orientasi kendaraan. Data posisi relatif terhadap jalan didapat dari mengukur jarak terdekat dari titik pusat masa kendaraan ke garis batas lajur kanan dan garis batas lajur kiri, dengan mengakses salah satu \textit{object} dari \textit{Bézier Path Creator}\cite{cit:16} \texttt{PathCreator.path} sehingga didapatkan \textit{object method} \texttt{GetClosestPointOnPath()} untuk mendapatkan koordinat terdekat dari mobil dari class tersebut, lalu akses class \textit{transform} dari \textit{game object} mobil, untuk mendapatkan koordinat dari \textit{center of mass} mobil simulator menggunakan \texttt{transform.position}, kemudian setelah didapatkan 2 koordinat yang ingin  diukur jaraknya, sehingga kemudian dapat menggunakan fungsi dari \textit{unity} yaitu \texttt{Vector3.Distance()} untuk mendapatkan jarak dari kedua koordinat tersebut. 

\par Kemudian Sudut Orientasi kendaraan pada pengujian ini menggunakan  Sudut spasial \textit{Euler} untuk mengetahui orientasi kendaraan pada 3 dimensi,  untuk mendapatkan nilai tersebut pada \textit{unity},  \textit{script} dapat mengakses \textit{class variable} \texttt{transform.eulerAngles}. 

\par  Maka dari itu, untuk posisi relatif, terdapat 2 macam data yaitu, Jarak dari batas kanan dan Jarak dari batas kiri dengan cara mengukur jarak dari 2 koordinat yaitu, \textit{center of mass} mobil dan titik terdekat dari \textit{center of mass} tersebut, sedangkan untuk Sudut Orientasi kendaraan, didapatkan 3 macam data yaitu, sudut orientasi relatif terhadap sumbu \textit{x,y,z} atau \textit{6 Degree of Freedom - Pitch, Yaw, and Roll}. Berikut datanya pada tabel \ref{tb:4_3}

Kesimpulan yang dapat ditarik pada pengujian pengambilan data informasi spasial adalah, pada tabel \ref{tb:4_3}, dapat diketahui bahwa lebar dari jalan pada simulator adalah kurang lebih 12 unit, dengan menggunakan 2 data tersebut, yaitu jarak mobil dari batas kiri dan kanan jalan, sehingga dapat diketahui lokasi tepatnya dari mobil. Selain itu, dapat disimpulkan juga bahwa mobil berada di jalan yang datar, dikarenakan nilai sudut \textit{euler} dari sumbu \textit{x} dan sumbu \textit{z} mendekati 0.

\section{Pengujian Pengambilan Data - \textit{Response Time} dan \textit{Input} Pengemudi}
\vspace{1ex}

\textit{Response Time} adalah waktu yang menunjukkan seberapa secepat pengemudi kembali ke lajur semula apabila simulator telah mendeteksi mobil keluar dari lajur sebelah kiri. Durasi yang dihitung adalah semenjak mobil keluar dari lajur, hingga mobil kembali ke lajur. Data yang didapat dari simulator berupa waktu dalam sekon, serta durasi frame semenjak mobil keluar dari lajur.

Selain itu diperlukannya perubahan definisi dari 'keluar jalur' apabila sistem mendeteksi bahwa pengemudi akan menyalip kendaraan yang lain. Pada kasus tersebut, pendeteksian keluar jalur akan diberikan \textit{flag} bahwasanya terdapat mobil didepan kendaraan pengemudi, sehingga pendeteksian sistem keluar jalur sementara dimatikan. Apabila sistem sudah mendeteksi tidak ada kendaraan didepan pengemudi, sistem pendeteksian keluar jalur akan berjalan kembali.

Kemudian, pada pengujian ini erat kaitannya dengan mendeteksi respon \textit{input} dari pengemudi. Maka dari itu, informasi \textit{input} dari pengemudi juga perlu di lakukan pengambilan datanya agar diketahui sudut dari \textit{steering wheel} serta tekanan pedal gas dan rem. Data - data \textit{input}ini penting dikarenakan berkaitan dengan data respon pengemudi ketika pengemudi mengetahui adanya rintangan atau objek tidak terduga ketika di jalan. Dengan menganalisa grafik sudut \textit{steering wheel} dan tekanan pedal gas dan rem, akan didapatkan redundansi data selain citra wajah pengemudi serta data - data biometrik pengemudi seperti EEG dan ECG, untuk analisa proses kapan terjadinya \textit{microsleep} pada pengemudi. Grafik data \textit{input} dari pengemudi bisa di lihat pada gambar \ref{fig:4.5}

\section{Pengujian Pengambilan Data - \textit{Citra Webcam}}
\vspace{1ex}

Pada gambar \ref{fig:4.2}, bisa dilihat hasil dari \textit{capture webcam} yang telah di \textit{encode} menjadi format \textit{.png}. 
\par Kesimpulan dari pengujian pengambilan data \textit{webcam} ini menunjukkan bahwa, harus dipastikan bahwa \textit{source} dari \textit{webcam} telah tepat terdeteksi sebagai suatu alat pengambil citra pada unity. Kemudian dipastikan apakah hasil \textit{encode} telah menghasilkan keluaran gambar yang diinginkan. Terjadinya kesalahan pendeteksian \textit{camera source} atau terjadinya kesalahan pada proses \textit{encode}, akan menyebabkan hasil keluaran citra \textit{webcam} tidak sesuai dengan yang diharapkan.

\section{Pengujian Pengambilan Data - \textit{Serial Data USB}}
\vspace{1ex}

\textit{Serial Data} disini ialah, data yang didapatkan dari komunikasi via \textit{serial USB}. Desain sistem komunikasi serial menggunakan port COM, bisa dilihat pada gambar \ref{tb:4_3}. 
\par Pada tugas akhir ini, sistem komunikasi serial arduino dengan - PC simulator adalah menggunakan script python untuk mengakses \textit{Port COM} yang menghubungkan Arduino dengan PC. Setelah \textit{Port COM} Diakses oleh script python, script python akan menerima \textit{byte} data dari Arduino, kemudian script tersebut akan merubah data menjadi angka bilangan bulat atau \textit{integer} tiap 4 \textit{byte (32 bit)} data yang masuk.
\par Kemudian setelah data dirubah menjadi data angka bilangan bulat, script python akan mengakses \textit{file system} dari komputer, dan akan membuat file dengan ekstensi \textit{.csv}, lalu menambahkan angka tersebut kedalam file yang baru dibuat tersebut, serta memformat dengan \textit{column separator ' \textbackslash{}t '} dan \textit{row separator ' \textbackslash{}n '}.

\section{Pengujian Respon Sinyal dari \textit{Steering Wheel Controller} terhadap simulator}
\vspace{1ex}

Pada pengujian ini, dikarenakan sistem memiliki mekanisme masukan dari pengguna berupa \textit{steering wheel controller}, maka diperlukan pengujian respon dari \textit{steering wheel controller} terhadap proses - proses berjalan di \textit{unity}. Kegiatan pengujian adalah dengan membandingkan sinyal dari \textit{steering wheel controller} yang berjalan sebelum \textit{script - script} lain pada unity, dengan sinyal yang setelah melalui proses - proses / \textit{script} lain pada unity. Hal ini dapat dilakukan dengan menerapkan fitur \textit{Script Execution Order} pada unity.
Grafik data sinyal tersebut bisa dilihat pada gambar \ref{fig:4.6} dan \ref{fig:4.7}, sedangkan grafik analisa \textit{error} kuantitatif nilai error bisa dilihat pada gambar \ref{fig:4.8}
\par Dari analisa kuantitaf terhadap nilai error tersebut, dapat dilihat bahwa grafik nilai error selalu mendekati 0 persen ketika \textit{runtime}, sehingga dapat disimpulkan bahwa \textit{input} dari \textit{steering wheel} tidak terlalu terpengaruh proses yang berjalan pada unity, sehingga grafik terlihat saling menindih.

\section{Pengujian \textit{User Experience} / UX dari Pengguna}
\vspace{1ex}

Pengujian kepuasan pengguna dilakukan dengan melakukan pengujian terhadap subjek pengendara mobil kemudian dilakukan proses pengisian kuesioner setelah pengguna mencoba alat simulator. Jumlah responden dalam pengujian kepuasan pengguna ini sebanyak 3 responden, terdiri dari keluarga dekat penulis. Hal ini dikarenakan keterbatasan situasi dan kondisi dimana ketika pengujian dilakukan sedang adanya pandemi \textit{COVID-19}, sehingga hal tersebut membatasi jumlah responden yang bersukarela mencoba alat simulator ini. Responden menerima 17 buah pernyataan sesuai pada tabel \ref{tb:4_7} Opsi tingkat persetujuan yang disediakan adalah sebagai berikut :

    \begin{enumerate}[nolistsep]
	\item Sangat Tidak Setuju (STS)
	\item Tidak Setuju (TS)
	\item Netral (N)
	\item Setuju (S)
	\item Sangat Setuju (SS)
	
	\vspace{1ex}
\end{enumerate}

%Daftar pertanyaan kuesioner UX
\begin{table}[]
\caption{Skenario kuesioner untuk pengujian kepuasan pengguna}
\label{tb:4_7}
\begin{tabular}{|c|p{9.5cm}|}
\hline
\textbf{No} & \multicolumn{1}{c|}{\textbf{Pertanyaan}}                                                                         \\ \hline
1           & Saya mengetahui tentang teknologi simulasi                                                                       \\ \hline
2           & Saya mengetahui tentang game yang melibatkan mengendarai mobil                                                   \\ \hline
3           & Saya pernah mengemudikan mobil virtual (game, simulator sim, dll)                                                \\ \hline
4           & Saya merasa menu User Interface Pemilihan Jumlah lajur aplikasi simulator   sudah jelas                          \\ \hline
5           & Saya merasa menu User Interface Pemilihan Jumlah lajur aplikasi simulator   sudah menarik                        \\ \hline
6           & Saya merasa user interface dashboard mobil sudah jelas                                                           \\ \hline
7           & Saya merasa user interface dashboard mobil sudah menarik                                                         \\ \hline
8           & Saya merasa user interface tutorial cara mengemudikan mobil sudah jelas                                          \\ \hline
9           & Saya merasa penggunaan perangkat keras steering wheel controller sudah   intuitif                                \\ \hline
10          & Saya merasa jumlah lajur yang yang dapat dipilih sudah mewakili berbagai   macam kondisi jalan yang sesungguhnya \\ \hline
11          & Saya merasa simulasi yang disajikan sudah cukup realistis                                                        \\ \hline
12          & Saya merasa pemandangan yang disediakan pada tiap - tiap lajur sudah   cukup menarik                             \\ \hline
13          & Saya merasa tertarik akan teknologi simulasi berkendara untuk riset                                              \\ \hline
14          & Saya merasa teknologi simulasi dapat menggantikan proses pengambilan data   di lapangan                          \\ \hline
15          & Saya menjadi tertarik untuk menggunakan simulator ini                                                            \\ \hline
16          & Saya setuju perangkat ini untuk diterapkan di institusi penelitian di   Indonesia                                \\ \hline
17          & Saya setuju perangkat ini dapat mendukung perkembangan riset deteksi   pengemudi mengantuk                       \\ \hline
\end{tabular}
\end{table}

Terdapat 3 macam pertanyaan yang diajukan, yaitu pertanyaan dengan tujuan untuk mengetahui bagaimana pengetahuan pengguna tentang teknologi simulasi khususnya teknogi simulasi berkendara, pertanyaan dengan tujuan untuk mengukur bagaimana kepuasan pengguna terhadap tampilan serta realitas simulasi, serta yang terakhir pertanyaan dengan tujuan untuk mengetahui bagaimana pendapat pengguna tentang teknologi simulator yang diterapkan untuk proses pengambilan data di lapangan.
\par Pengujian kepuasan pengguna dihitung dari banyaknya presentase total poin yang didapatkan dari total maksimal poin untuk setiap pernyataan yang disediakan (likert scale). Berdasarkan parameter tersebut, setiap opsi memiliki poin yang berbeda yaitu SS=5 poin, S=4 poin, N=3 poin, TS=2 poin, dan STS=1 poin. Semakin besar presentasenya, maka semakin \textit{valid} pernyataan tersebut. 
\par Berikut hasil dari proses pengujian tingkat kepuasan pengguna (tabel \ref{tb:4_8})

\begin{table}[]
\caption{Hasil pengujian kepuasan pengguna.}
\label{tb:4_8}
\begin{tabular}{|c|l|l|l|l|l|}
\hline
\multirow{2}{*}{\textbf{Pernyataan}} & \multicolumn{5}{c|}{\textbf{Jawaban}}                                                                                                                                       \\ \cline{2-6} 
                                     & \multicolumn{1}{c|}{\textbf{STS}} & \multicolumn{1}{c|}{\textbf{TS}} & \multicolumn{1}{c|}{\textbf{N}} & \multicolumn{1}{c|}{\textbf{S}} & \multicolumn{1}{c|}{\textbf{SS}} \\ \hline
Pernyataan 1                         & 0,00\%                            & 0,00\%                           & 0,00\%                          & 66,67\%                         & 33,33\%                          \\ \hline
Pernyataan 2                         & 33,33\%                           & 0,00\%                           & 0,00\%                          & 33,33\%                         & 33,33\%                          \\ \hline
Pernyataan 3                         & 33,33\%                           & 0,00\%                           & 0,00\%                          & 33,33\%                         & 33,33\%                          \\ \hline
Pernyataan 4                         & 0,00\%                            & 0,00\%                           & 33,33\%                         & 33,33\%                         & 33,33\%                          \\ \hline
Pernyataan 5                         & 0,00\%                            & 0,00\%                           & 0,00\%                          & 100,00\%                        & 0,00\%                           \\ \hline
Pernyataan 6                         & 0,00\%                            & 0,00\%                           & 33,33\%                         & 33,33\%                         & 33,33\%                          \\ \hline
Pernyataan 7                         & 0,00\%                            & 0,00\%                           & 33,33\%                         & 66,67\%                         & 0,00\%                           \\ \hline
Pernyataan 8                         & 0,00\%                            & 0,00\%                           & 0,00\%                          & 33,33\%                         & 66,67\%                          \\ \hline
Pernyataan 9                         & 0,00\%                            & 0,00\%                           & 100,00\%                        & 0,00\%                          & 0,00\%                           \\ \hline
Pernyataan 10                        & 0,00\%                            & 0,00\%                           & 33,33\%                         & 33,33\%                         & 33,33\%                          \\ \hline
Pernyataan 11                        & 0,00\%                            & 0,00\%                           & 0,00\%                          & 33,33\%                         & 66,67\%                          \\ \hline
Pernyataan 12                        & 0,00\%                            & 0,00\%                           & 33,33\%                         & 33,33\%                         & 33,33\%                          \\ \hline
Pernyataan 13                        & 0,00\%                            & 0,00\%                           & 0,00\%                          & 33,33\%                         & 66,67\%                          \\ \hline
Pernyataan 14                        & 0,00\%                            & 0,00\%                           & 33,33\%                         & 33,33\%                         & 33,33\%                          \\ \hline
Pernyataan 15                        & 0,00\%                            & 0,00\%                           & 66,67\%                         & 0,00\%                          & 33,33\%                          \\ \hline
Pernyataan 16                        & 0,00\%                            & 0,00\%                           & 0,00\%                          & 33,33\%                         & 66,67\%                          \\ \hline
Pernyataan 17                        & 0,00\%                            & 0,00\%                           & 0,00\%                          & 33,33\%                         & 66,67\%                          \\ \hline
\end{tabular}
\end{table}

% GAMBAR USER INTERFACE
\begin{figure} [!htb]
	\captionsetup{justification=centering}
	\includegraphics[scale=0.1]{img/UItest.jpg}
	\caption{Korelasi \textit{user interface} pemilihan lajur dengan \textit{scene} yang dimuat oleh simulator}
	\label{fig:4.1}
\end{figure}

%DATA SCREENSHOT WEBCAM
\begin{figure} [!htb]
	\captionsetup{justification=centering}
	\includegraphics[scale=0.23]{img/webcam.JPG}
	\caption{\textit{Frame webcam} yang tersimpan di dalam \textit{harddrive}}
	\label{fig:4.2}
\end{figure}

%DATA GRAFIK SENSOR SERIAL
\begin{figure} [!htb]
	\captionsetup{justification=centering}
	\includegraphics[scale=0.4]{img/arduino-python.JPG}
	\caption{Diagram Sistem Komunikasi Arduino Dengan PC Simulator}
	\label{fig:4.3}
\end{figure}

\begin{figure} [!htb]
	\captionsetup{justification=centering}
	\includegraphics[scale=0.4]{img/serial.JPG}
	\caption{Contoh data serial yang diterima oleh arduino}
	\label{fig:4.4}
\end{figure}

\begin{figure} [!htb]
	\captionsetup{justification=centering}
	\includegraphics[scale=0.3]{img/driverinput.JPG}
	\caption{\textit{Input} dari pengemudi - \textit{Horizontal Axis} adalah Data sudut \textit{steering wheel}, \textit{Vertical Axis} adalah data nilai tekanan pedal gas / rem}
	\label{fig:4.5}
\end{figure}

\begin{figure} [!htb]
	\captionsetup{justification=centering}
	\includegraphics[scale=0.3]{img/response_horz.JPG}
	\caption{Perbandingan sinyal dari \textit{steering wheel controller}, sebelum dan sesudah grafik unity - \textit{Horizontal Axis} / Sudut \textit{steering wheel}}
	\label{fig:4.6}
\end{figure}

\begin{figure} [!htb]
	\captionsetup{justification=centering}
	\includegraphics[scale=0.3]{img/response_vert.JPG}
	\caption{Perbandingan sinyal dari \textit{steering wheel controller}, sebelum dan sesudah grafik unity - \textit{Vertical Axis} / Nilai tekanan pedal gas/rem}
	\label{fig:4.7}
\end{figure}

\begin{figure} [!htb]
	\captionsetup{justification=centering}
	\includegraphics[scale=0.6]{img/grafik-error.JPG}
	\caption{Grafik analisa kuantitatif nilai error dari \textit{steering wheel controller}}
	\label{fig:4.8}
\end{figure}

% TABEL INPUT PENGEMUDI
\begin{table}[]
\caption{Tabel \textit{Input} Pengemudi Selama Kurang Lebih 1 Detik}
\label{tb:4_6}
\begin{tabular}{|c|c|c|c|}
\hline
\textit{\textbf{HorizontalAxis}} & \textit{\textbf{VerticalAxis}} & \textit{\textbf{Frame}} & \textit{\textbf{DateTime}} \\ \hline
0                                & 0                              & 74                      & 01.22.25.590387            \\ \hline
0                                & 0,8822352                      & 75                      & 01.22.25.748759            \\ \hline
0                                & 1                              & 76                      & 01.22.25.923766            \\ \hline
0                                & 1                              & 77                      & 01.22.26.234632            \\ \hline
0                                & 1                              & 78                      & 01.22.26.413535            \\ \hline
0                                & 1                              & 79                      & 01.22.26.570931            \\ \hline
0                                & 1                              & 80                      & 01.22.26.736147            \\ \hline
0                                & 1                              & 81                      & 01.22.26.892566            \\ \hline
0                                & 1                              & 82                      & 01.22.27.071471            \\ \hline
0                                & 1                              & 83                      & 01.22.27.225934            \\ \hline
0                                & 1                              & 84                      & 01.22.27.390172            \\ \hline
-0,007369441                     & 1                              & 85                      & 01.22.27.574942            \\ \hline
-0,08203134                      & 1                              & 86                      & 01.22.27.756776            \\ \hline
-0,08203134                      & 1                              & 87                      & 01.22.27.921995            \\ \hline
-0,08203134                      & 1                              & 88                      & 01.22.28.186928            \\ \hline
-0,08203134                      & 1                              & 89                      & 01.22.28.361919            \\ \hline
-0,08203134                      & 1                              & 90                      & 01.22.28.542779            \\ \hline
-0,03229748                      & 1                              & 91                      & 01.22.28.699196            \\ \hline
-0,00187061                      & 1                              & 92                      & 01.22.28.865390            \\ \hline
0                                & 1                              & 93                      & 01.22.29.040383            \\ \hline
0                                & 1                              & 94                      & 01.22.29.211465            \\ \hline
0                                & 1                              & 95                      & 01.22.29.367885            \\ \hline
0,001748414                      & 1                              & 96                      & 01.22.29.528212            \\ \hline
0,06535155                       & 1                              & 97                      & 01.22.29.686586            \\ \hline
0,1510722                        & 1                              & 98                      & 01.22.30.072742            \\ \hline
0,3418816                        & 1                              & 99                      & 01.22.30.228184            \\ \hline
0,4662774                        & 1                              & 100                     & 01.22.30.431527            \\ \hline
0,5492486                        & 1                              & 101                     & 01.22.30.600656            \\ \hline
0,6156012                        & 1                              & 102                     & 01.22.30.783469            \\ \hline
0,6681455                        & 1                              & 103                     & 01.22.30.942821            \\ \hline
0,7344981                        & 1                              & 104                     & 01.22.31.153007            \\ \hline
0,8036612                        & 1                              & 105                     & 01.22.31.317244            \\ \hline
0,8700138                        & 1                              & 106                     & 01.22.31.499081            \\ \hline
0,9363663                        & 1                              & 107                     & 01.22.31.668210            \\ \hline
0,972292                         & 1                              & 108                     & 01.22.31.833426            \\ \hline
\end{tabular}
\end{table}

% TABEL KECEPATAN
\begin{table}[]
\caption{Data Informasi Kecepatan}
\label{tb:4_2}
\begin{tabular}{|c|l|l|l|l|c|}
\hline
                               & \multicolumn{1}{c|}{\textit{VelX}} & \multicolumn{1}{c|}{\textit{VelY}} & \multicolumn{1}{c|}{\textit{VelZ}} & \multicolumn{1}{c|}{\textit{VelMagn}} & \textit{AvgFrameRate}          \\ \cline{2-6} 
\multirow{-2}{*}{\textit{No.}} & \multicolumn{4}{c|}{\textit{(Unit/Frame)}}                                                                                                           & \textit{(Frame/Sec)}           \\ \hline
{\color[HTML]{000000} 1}       & {\color[HTML]{000000} 1,0,E-05}    & {\color[HTML]{000000} -1,1,E-04}   & {\color[HTML]{000000} 4,4,E-06}    & {\color[HTML]{000000} 1,1,E-04}       & {\color[HTML]{000000} 52,2394} \\ \hline
{\color[HTML]{000000} 2}       & {\color[HTML]{000000} 1,1,E-05}    & {\color[HTML]{000000} -9,9,E-05}   & {\color[HTML]{000000} 3,7,E-06}    & {\color[HTML]{000000} 9,9,E-05}       & {\color[HTML]{000000} 52,2469} \\ \hline
{\color[HTML]{000000} 3}       & {\color[HTML]{000000} 1,2,E-05}    & {\color[HTML]{000000} -8,3,E-05}   & {\color[HTML]{000000} 3,3,E-06}    & {\color[HTML]{000000} 8,4,E-05}       & {\color[HTML]{000000} 52,2751} \\ \hline
{\color[HTML]{000000} 4}       & {\color[HTML]{000000} 1,2,E-05}    & {\color[HTML]{000000} -6,5,E-05}   & {\color[HTML]{000000} 2,8,E-06}    & {\color[HTML]{000000} 6,6,E-05}       & {\color[HTML]{000000} 52,2905} \\ \hline
{\color[HTML]{000000} 5}       & {\color[HTML]{000000} 1,2,E-05}    & {\color[HTML]{000000} -4,5,E-05}   & {\color[HTML]{000000} 2,3,E-06}    & {\color[HTML]{000000} 4,7,E-05}       & {\color[HTML]{000000} 52,3272} \\ \hline
{\color[HTML]{000000} 6}       & {\color[HTML]{000000} 1,1,E-05}    & {\color[HTML]{000000} -2,6,E-05}   & {\color[HTML]{000000} 1,9,E-06}    & {\color[HTML]{000000} 2,8,E-05}       & {\color[HTML]{000000} 52,3012} \\ \hline
{\color[HTML]{000000} 7}       & {\color[HTML]{000000} 1,0,E-05}    & {\color[HTML]{000000} -5,8,E-06}   & {\color[HTML]{000000} 1,6,E-06}    & {\color[HTML]{000000} 1,2,E-05}       & {\color[HTML]{000000} 52,3211} \\ \hline
{\color[HTML]{000000} 8}       & {\color[HTML]{000000} 8,6,E-06}    & {\color[HTML]{000000} 1,2,E-05}    & {\color[HTML]{000000} 1,3,E-06}    & {\color[HTML]{000000} 1,5,E-05}       & {\color[HTML]{000000} 52,3379} \\ \hline
{\color[HTML]{000000} 9}       & {\color[HTML]{000000} 7,1,E-06}    & {\color[HTML]{000000} 2,8,E-05}    & {\color[HTML]{000000} 1,1,E-06}    & {\color[HTML]{000000} 2,9,E-05}       & {\color[HTML]{000000} 52,3672} \\ \hline
{\color[HTML]{000000} 10}      & {\color[HTML]{000000} 5,4,E-06}    & {\color[HTML]{000000} 4,2,E-05}    & {\color[HTML]{000000} 9,2,E-07}    & {\color[HTML]{000000} 4,2,E-05}       & {\color[HTML]{000000} 52,3583} \\ \hline
{\color[HTML]{000000} 11}      & {\color[HTML]{000000} 3,7,E-06}    & {\color[HTML]{000000} 5,3,E-05}    & {\color[HTML]{000000} 5,0,E-07}    & {\color[HTML]{000000} 5,3,E-05}       & {\color[HTML]{000000} 52,3787} \\ \hline
{\color[HTML]{000000} 12}      & {\color[HTML]{000000} 3,7,E-06}    & {\color[HTML]{000000} 5,3,E-05}    & {\color[HTML]{000000} 5,0,E-07}    & {\color[HTML]{000000} 5,3,E-05}       & {\color[HTML]{000000} 52,4137} \\ \hline
{\color[HTML]{000000} 13}      & {\color[HTML]{000000} 2,1,E-06}    & {\color[HTML]{000000} 6,0,E-05}    & {\color[HTML]{000000} 1,6,E-07}    & {\color[HTML]{000000} 6,1,E-05}       & {\color[HTML]{000000} 52,4484} \\ \hline
{\color[HTML]{000000} 14}      & {\color[HTML]{000000} 6,2,E-07}    & {\color[HTML]{000000} 6,5,E-05}    & {\color[HTML]{000000} -4,1,E-07}   & {\color[HTML]{000000} 6,5,E-05}       & {\color[HTML]{000000} 52,4828} \\ \hline
{\color[HTML]{000000} 15}      & {\color[HTML]{000000} -4,7,E-07}   & {\color[HTML]{000000} 6,6,E-05}    & {\color[HTML]{000000} -9,2,E-07}   & {\color[HTML]{000000} 6,6,E-05}       & {\color[HTML]{000000} 52,4952} \\ \hline
{\color[HTML]{000000} 16}      & {\color[HTML]{000000} -1,3,E-06}   & {\color[HTML]{000000} 6,5,E-05}    & {\color[HTML]{000000} -1,3,E-06}   & {\color[HTML]{000000} 6,5,E-05}       & {\color[HTML]{000000} 52,5117} \\ \hline
{\color[HTML]{000000} 17}      & {\color[HTML]{000000} -1,9,E-06}   & {\color[HTML]{000000} 6,1,E-05}    & {\color[HTML]{000000} -1,8,E-06}   & {\color[HTML]{000000} 6,1,E-05}       & {\color[HTML]{000000} 52,5453} \\ \hline
{\color[HTML]{000000} 18}      & {\color[HTML]{000000} -2,1,E-06}   & {\color[HTML]{000000} 5,5,E-05}    & {\color[HTML]{000000} -1,8,E-06}   & {\color[HTML]{000000} 5,5,E-05}       & {\color[HTML]{000000} 52,5479} \\ \hline
{\color[HTML]{000000} 19}      & {\color[HTML]{000000} -2,1,E-06}   & {\color[HTML]{000000} 5,5,E-05}    & {\color[HTML]{000000} -1,8,E-06}   & {\color[HTML]{000000} 5,5,E-05}       & {\color[HTML]{000000} 52,5684} \\ \hline
{\color[HTML]{000000} 20}      & {\color[HTML]{000000} -2,3,E-06}   & {\color[HTML]{000000} 4,7,E-05}    & {\color[HTML]{000000} -2,4,E-06}   & {\color[HTML]{000000} 4,8,E-05}       & {\color[HTML]{000000} 52,6013} \\ \hline
{\color[HTML]{000000} 21}      & {\color[HTML]{000000} -2,3,E-06}   & {\color[HTML]{000000} 3,9,E-05}    & {\color[HTML]{000000} -2,5,E-06}   & {\color[HTML]{000000} 3,9,E-05}       & {\color[HTML]{000000} 52,6340} \\ \hline
{\color[HTML]{000000} 22}      & {\color[HTML]{000000} -2,2,E-06}   & {\color[HTML]{000000} 2,8,E-05}    & {\color[HTML]{000000} -2,2,E-06}   & {\color[HTML]{000000} 2,9,E-05}       & {\color[HTML]{000000} 52,6664} \\ \hline
{\color[HTML]{000000} 23}      & {\color[HTML]{000000} -2,1,E-06}   & {\color[HTML]{000000} 1,7,E-05}    & {\color[HTML]{000000} -1,8,E-06}   & {\color[HTML]{000000} 1,8,E-05}       & {\color[HTML]{000000} 52,6986} \\ \hline
{\color[HTML]{000000} 24}      & {\color[HTML]{000000} -2,1,E-06}   & {\color[HTML]{000000} 7,0,E-06}    & {\color[HTML]{000000} -1,7,E-06}   & {\color[HTML]{000000} 7,5,E-06}       & {\color[HTML]{000000} 52,7100} \\ \hline
{\color[HTML]{000000} 25}      & {\color[HTML]{000000} -2,0,E-06}   & {\color[HTML]{000000} -2,4,E-06}   & {\color[HTML]{000000} -1,2,E-06}   & {\color[HTML]{000000} 3,4,E-06}       & {\color[HTML]{000000} 52,6537} \\ \hline
{\color[HTML]{000000} 26}      & {\color[HTML]{000000} -2,0,E-06}   & {\color[HTML]{000000} -1,1,E-05}   & {\color[HTML]{000000} -8,4,E-07}   & {\color[HTML]{000000} 1,1,E-05}       & {\color[HTML]{000000} 52,6855} \\ \hline
{\color[HTML]{000000} 27}      & {\color[HTML]{000000} -1,9,E-06}   & {\color[HTML]{000000} -1,8,E-05}   & {\color[HTML]{000000} -1,1,E-06}   & {\color[HTML]{000000} 1,9,E-05}       & {\color[HTML]{000000} 52,7169} \\ \hline
{\color[HTML]{000000} 28}      & {\color[HTML]{000000} -1,9,E-06}   & {\color[HTML]{000000} -1,8,E-05}   & {\color[HTML]{000000} -1,1,E-06}   & {\color[HTML]{000000} 1,9,E-05}       & {\color[HTML]{000000} 52,7482} \\ \hline
{\color[HTML]{000000} 29}      & {\color[HTML]{000000} -1,9,E-06}   & {\color[HTML]{000000} -2,5,E-05}   & {\color[HTML]{000000} -5,4,E-07}   & {\color[HTML]{000000} 2,5,E-05}       & {\color[HTML]{000000} 52,7733} \\ \hline
{\color[HTML]{000000} 30}      & {\color[HTML]{000000} -1,9,E-06}   & {\color[HTML]{000000} -2,5,E-05}   & {\color[HTML]{000000} -5,4,E-07}   & {\color[HTML]{000000} 2,5,E-05}       & {\color[HTML]{000000} 52,7733} \\ \hline
\end{tabular}
\end{table}
\vspace{1ex}

% TABEL INFORMASI SPASIAL
\begin{table}[!htb]
\centering
\caption{Data Informasi Spasial}
\label{tb:4_3}
\begin{tabular}{|c|l|l|l|l|l|}
\hline
                               & \multicolumn{1}{c|}{}                                                                                       & \multicolumn{1}{c|}{}                                                                                        & \multicolumn{3}{c|}{\textit{\begin{tabular}[c]{@{}c@{}}Euler Angles\\ (Degrees)\end{tabular}}}      \\ \cline{4-6} 
\multirow{-2}{*}{\textit{No.}} & \multicolumn{1}{c|}{\multirow{-2}{*}{\textit{\begin{tabular}[c]{@{}c@{}}Dist. Left\\ (Unit)\end{tabular}}}} & \multicolumn{1}{c|}{\multirow{-2}{*}{\textit{\begin{tabular}[c]{@{}c@{}}Dist. Right\\ (Unit)\end{tabular}}}} & \multicolumn{1}{c|}{\textit{pitch}} & \multicolumn{1}{c|}{\textit{yaw}} & \multicolumn{1}{c|}{\textit{roll}} \\ \hline
1                              & {\color[HTML]{000000} 4,290802}                                                                             & {\color[HTML]{000000} 7,87863}                                                                               & {\color[HTML]{000000} (0,0,}    & {\color[HTML]{000000} 91,7,}    & {\color[HTML]{000000} 0,0)}     \\ \hline
2                              & {\color[HTML]{000000} 4,290906}                                                                             & {\color[HTML]{000000} 7,878543}                                                                              & {\color[HTML]{000000} (0,1,}    & {\color[HTML]{000000} 91,7,}    & {\color[HTML]{000000} 0,0)}     \\ \hline
3                              & {\color[HTML]{000000} 4,290405}                                                                             & {\color[HTML]{000000} 7,879219}                                                                              & {\color[HTML]{000000} (0,5,}    & {\color[HTML]{000000} 91,7,}    & {\color[HTML]{000000} 0,0)}     \\ \hline
4                              & {\color[HTML]{000000} 4,290458}                                                                             & {\color[HTML]{000000} 7,879376}                                                                              & {\color[HTML]{000000} (0,7,}    & {\color[HTML]{000000} 91,7,}    & {\color[HTML]{000000} 0,0)}     \\ \hline
5                              & {\color[HTML]{000000} 4,290488}                                                                             & {\color[HTML]{000000} 7,879422}                                                                              & {\color[HTML]{000000} (0,7,}    & {\color[HTML]{000000} 91,7,}    & {\color[HTML]{000000} 0,0)}     \\ \hline
6                              & {\color[HTML]{000000} 4,29053}                                                                              & {\color[HTML]{000000} 7,879436}                                                                              & {\color[HTML]{000000} (0,7,}    & {\color[HTML]{000000} 91,7,}    & {\color[HTML]{000000} 0,0)}     \\ \hline
7                              & {\color[HTML]{000000} 4,29057}                                                                              & {\color[HTML]{000000} 7,879446}                                                                              & {\color[HTML]{000000} (0,6,}    & {\color[HTML]{000000} 91,7,}    & {\color[HTML]{000000} 0,0)}     \\ \hline
8                              & {\color[HTML]{000000} 4,290622}                                                                             & {\color[HTML]{000000} 7,879434}                                                                              & {\color[HTML]{000000} (0,6,}    & {\color[HTML]{000000} 91,7,}    & {\color[HTML]{000000} 0,0)}     \\ \hline
9                              & {\color[HTML]{000000} 4,290658}                                                                             & {\color[HTML]{000000} 7,879415}                                                                              & {\color[HTML]{000000} (0,5,}    & {\color[HTML]{000000} 91,7,}    & {\color[HTML]{000000} 0,0)}     \\ \hline
10                             & {\color[HTML]{000000} 4,290717}                                                                             & {\color[HTML]{000000} 7,879354}                                                                              & {\color[HTML]{000000} (0,4,}    & {\color[HTML]{000000} 91,7,}    & {\color[HTML]{000000} 0,0)}     \\ \hline
11                             & {\color[HTML]{000000} 4,290747}                                                                             & {\color[HTML]{000000} 7,87931}                                                                               & {\color[HTML]{000000} (0,4,}    & {\color[HTML]{000000} 91,7,}    & {\color[HTML]{000000} 0,0)}     \\ \hline
12                             & {\color[HTML]{000000} 4,290762}                                                                             & {\color[HTML]{000000} 7,879263}                                                                              & {\color[HTML]{000000} (0,3,}    & {\color[HTML]{000000} 91,7,}    & {\color[HTML]{000000} 0,0)}     \\ \hline
13                             & {\color[HTML]{000000} 4,290767}                                                                             & {\color[HTML]{000000} 7,879231}                                                                              & {\color[HTML]{000000} (0,3,}    & {\color[HTML]{000000} 91,7,}    & {\color[HTML]{000000} 0,0)}     \\ \hline
14                             & {\color[HTML]{000000} 4,290767}                                                                             & {\color[HTML]{000000} 7,879195}                                                                              & {\color[HTML]{000000} (0,3,}    & {\color[HTML]{000000} 91,7,}    & {\color[HTML]{000000} 0,0)}     \\ \hline
15                             & {\color[HTML]{000000} 4,29076}                                                                              & {\color[HTML]{000000} 7,879167}                                                                              & {\color[HTML]{000000} (0,2,}    & {\color[HTML]{000000} 91,7,}    & {\color[HTML]{000000} 0,0)}     \\ \hline
16                             & {\color[HTML]{000000} 4,290745}                                                                             & {\color[HTML]{000000} 7,879142}                                                                              & {\color[HTML]{000000} (0,2,}    & {\color[HTML]{000000} 91,7,}    & {\color[HTML]{000000} 0,0)}     \\ \hline
17                             & {\color[HTML]{000000} 4,290729}                                                                             & {\color[HTML]{000000} 7,879126}                                                                              & {\color[HTML]{000000} (0,2,}    & {\color[HTML]{000000} 91,7,}    & {\color[HTML]{000000} 0,0)}     \\ \hline
18                             & {\color[HTML]{000000} 4,290688}                                                                             & {\color[HTML]{000000} 7,879111}                                                                              & {\color[HTML]{000000} (0,2,}    & {\color[HTML]{000000} 91,7,}    & {\color[HTML]{000000} 0,0)}     \\ \hline
19                             & {\color[HTML]{000000} 4,290656}                                                                             & {\color[HTML]{000000} 7,879109}                                                                              & {\color[HTML]{000000} (0,3,}    & {\color[HTML]{000000} 91,7,}    & {\color[HTML]{000000} 0,0)}     \\ \hline
20                             & {\color[HTML]{000000} 4,290623}                                                                             & {\color[HTML]{000000} 7,879117}                                                                              & {\color[HTML]{000000} (0,3,}    & {\color[HTML]{000000} 91,7,}    & {\color[HTML]{000000} 0,0)}     \\ \hline
21                             & {\color[HTML]{000000} 4,290589}                                                                             & {\color[HTML]{000000} 7,879128}                                                                              & {\color[HTML]{000000} (0,3,}    & {\color[HTML]{000000} 91,7,}    & {\color[HTML]{000000} 0,0)}     \\ \hline
22                             & {\color[HTML]{000000} 4,290563}                                                                             & {\color[HTML]{000000} 7,879138}                                                                              & {\color[HTML]{000000} (0,3,}    & {\color[HTML]{000000} 91,7,}    & {\color[HTML]{000000} 0,0)}     \\ \hline
23                             & {\color[HTML]{000000} 4,290539}                                                                             & {\color[HTML]{000000} 7,879152}                                                                              & {\color[HTML]{000000} (0,4,}    & {\color[HTML]{000000} 91,7,}    & {\color[HTML]{000000} 0,0)}     \\ \hline
24                             & {\color[HTML]{000000} 4,290514}                                                                             & {\color[HTML]{000000} 7,87917}                                                                               & {\color[HTML]{000000} (0,4,}    & {\color[HTML]{000000} 91,7,}    & {\color[HTML]{000000} 0,0)}     \\ \hline
25                             & {\color[HTML]{000000} 4,290488}                                                                             & {\color[HTML]{000000} 7,879189}                                                                              & {\color[HTML]{000000} (0,4,}    & {\color[HTML]{000000} 91,7,}    & {\color[HTML]{000000} 0,0)}     \\ \hline
26                             & {\color[HTML]{000000} 4,290459}                                                                             & {\color[HTML]{000000} 7,879222}                                                                              & {\color[HTML]{000000} (0,5,}    & {\color[HTML]{000000} 91,7,}    & {\color[HTML]{000000} 0,0)}     \\ \hline
27                             & {\color[HTML]{000000} 4,290454}                                                                             & {\color[HTML]{000000} 7,879234}                                                                              & {\color[HTML]{000000} (0,5,}    & {\color[HTML]{000000} 91,7,}    & {\color[HTML]{000000} 0,0)}     \\ \hline
28                             & {\color[HTML]{000000} 4,290452}                                                                             & {\color[HTML]{000000} 7,879245}                                                                              & {\color[HTML]{000000} (0,5,}    & {\color[HTML]{000000} 91,7,}    & {\color[HTML]{000000} 0,0)}     \\ \hline
29                             & {\color[HTML]{000000} 4,290462}                                                                             & {\color[HTML]{000000} 7,87925}                                                                               & {\color[HTML]{000000} (0,5,}    & {\color[HTML]{000000} 91,7,}    & {\color[HTML]{000000} 0,0)}     \\ \hline
\end{tabular}
\end{table}
\vspace{1ex}

% TABEL RESPONSE TIME
\begin{table}[]
\centering
\caption{Tabel \textit{Response Time}}
\label{tb:4_4}
\begin{tabular}{|c|c|c|c|c|}
\hline
{\color[HTML]{000000} }                               & {\color[HTML]{000000} }                                 & {\color[HTML]{000000} }                                  & \multicolumn{2}{c|}{{\color[HTML]{000000} \textit{Duration}}}                   \\ \cline{4-5} 
\multirow{-2}{*}{{\color[HTML]{000000} \textit{No.}}} & \multirow{-2}{*}{{\color[HTML]{000000} \textit{Start}}} & \multirow{-2}{*}{{\color[HTML]{000000} \textit{Return}}} & {\color[HTML]{000000} \textit{Seconds}} & {\color[HTML]{000000} \textit{Frame}} \\ \hline
{\color[HTML]{000000} 1}                              & {\color[HTML]{000000} 1:00:37}                          & {\color[HTML]{000000} 1:00:43}                           & {\color[HTML]{000000} 6,707}            & {\color[HTML]{000000} 366,538}        \\ \hline
{\color[HTML]{000000} 2}                              & {\color[HTML]{000000} 1:03:15}                          & {\color[HTML]{000000} 1:03:16}                           & {\color[HTML]{000000} 1,542}            & {\color[HTML]{000000} 84,517}         \\ \hline
{\color[HTML]{000000} 3}                              & {\color[HTML]{000000} 1:06:21}                          & {\color[HTML]{000000} 1:06:30}                           & {\color[HTML]{000000} 9,048}            & {\color[HTML]{000000} 488,140}        \\ \hline
{\color[HTML]{000000} 4}                              & {\color[HTML]{000000} 1:08:05}                          & {\color[HTML]{000000} 1:08:08}                           & {\color[HTML]{000000} 3,141}            & {\color[HTML]{000000} 177,906}        \\ \hline
{\color[HTML]{000000} 5}                              & {\color[HTML]{000000} 1:09:58}                          & {\color[HTML]{000000} 1:10:07}                           & {\color[HTML]{000000} 9,863}            & {\color[HTML]{000000} 594,838}        \\ \hline
{\color[HTML]{000000} 6}                              & {\color[HTML]{000000} 1:10:19}                          & {\color[HTML]{000000} 1:10:29}                           & {\color[HTML]{000000} 0,878}            & {\color[HTML]{000000} 50,819}         \\ \hline
{\color[HTML]{000000} 7}                              & {\color[HTML]{000000} 1:12:47}                          & {\color[HTML]{000000} 1:12:48}                           & {\color[HTML]{000000} 1,936}            & {\color[HTML]{000000} 111,184}        \\ \hline
{\color[HTML]{000000} 8}                              & {\color[HTML]{000000} 1:13:00}                          & {\color[HTML]{000000} 1:13:00}                           & {\color[HTML]{000000} 0,947}            & {\color[HTML]{000000} 54,926}         \\ \hline
{\color[HTML]{000000} 9}                              & {\color[HTML]{000000} 1:14:27}                          & {\color[HTML]{000000} 1:14:31}                           & {\color[HTML]{000000} 4,245}            & {\color[HTML]{000000} 235,470}        \\ \hline
{\color[HTML]{000000} 10}                             & {\color[HTML]{000000} 1:16:25}                          & {\color[HTML]{000000} 1:16:34}                           & {\color[HTML]{000000} 9,917}            & {\color[HTML]{000000} 579,450}        \\ \hline
{\color[HTML]{000000} 11}                             & {\color[HTML]{000000} 1:19:26}                          & {\color[HTML]{000000} 1:19:27}                           & {\color[HTML]{000000} 1,972}            & {\color[HTML]{000000} 116,880}        \\ \hline
{\color[HTML]{000000} 12}                             & {\color[HTML]{000000} 1:20:05}                          & {\color[HTML]{000000} 1:20:05}                           & {\color[HTML]{000000} 0,266}            & {\color[HTML]{000000} 14,058}         \\ \hline
{\color[HTML]{000000} 13}                             & {\color[HTML]{000000} 1:21:40}                          & {\color[HTML]{000000} 1:21:50}                           & {\color[HTML]{000000} 0,911}            & {\color[HTML]{000000} 56,291}         \\ \hline
{\color[HTML]{000000} 14}                             & {\color[HTML]{000000} 1:27:39}                          & {\color[HTML]{000000} 1:27:49}                           & {\color[HTML]{000000} 0,734}            & {\color[HTML]{000000} 42,888}         \\ \hline
{\color[HTML]{000000} 15}                             & {\color[HTML]{000000} 1:36:56}                          & {\color[HTML]{000000} 1:36:56}                           & {\color[HTML]{000000} 0,972}            & {\color[HTML]{000000} 53,664}         \\ \hline
{\color[HTML]{000000} 16}                             & {\color[HTML]{000000} 1:37:04}                          & {\color[HTML]{000000} 1:37:12}                           & {\color[HTML]{000000} 8,662}            & {\color[HTML]{000000} 475,197}        \\ \hline
{\color[HTML]{000000} 17}                             & {\color[HTML]{000000} 1:38:35}                          & {\color[HTML]{000000} 1:38:40}                           & {\color[HTML]{000000} 5,015}            & {\color[HTML]{000000} 280,840}        \\ \hline
{\color[HTML]{000000} 18}                             & {\color[HTML]{000000} 1:39:44}                          & {\color[HTML]{000000} 1:39:49}                           & {\color[HTML]{000000} 5,885}            & {\color[HTML]{000000} 307,903}        \\ \hline
{\color[HTML]{000000} 19}                             & {\color[HTML]{000000} 1:40:17}                          & {\color[HTML]{000000} 1:40:22}                           & {\color[HTML]{000000} 5,361}            & {\color[HTML]{000000} 320,641}        \\ \hline
{\color[HTML]{000000} 20}                             & {\color[HTML]{000000} 1:43:05}                          & {\color[HTML]{000000} 1:43:09}                           & {\color[HTML]{000000} 4,009}            & {\color[HTML]{000000} 243,587}        \\ \hline
{\color[HTML]{000000} 21}                             & {\color[HTML]{000000} 1:45:51}                          & {\color[HTML]{000000} 1:45:53}                           & {\color[HTML]{000000} 2,657}            & {\color[HTML]{000000} 150,546}        \\ \hline
{\color[HTML]{000000} 22}                             & {\color[HTML]{000000} 1:48:19}                          & {\color[HTML]{000000} 1:48:23}                           & {\color[HTML]{000000} 4,068}            & {\color[HTML]{000000} 220,445}        \\ \hline
{\color[HTML]{000000} 23}                             & {\color[HTML]{000000} 1:48:27}                          & {\color[HTML]{000000} 1:48:30}                           & {\color[HTML]{000000} 3,068}            & {\color[HTML]{000000} 168,034}        \\ \hline
{\color[HTML]{000000} 24}                             & {\color[HTML]{000000} 1:51:09}                          & {\color[HTML]{000000} 1:51:17}                           & {\color[HTML]{000000} 8,246}            & {\color[HTML]{000000} 447,923}        \\ \hline
{\color[HTML]{000000} 25}                             & {\color[HTML]{000000} 1:55:19}                          & {\color[HTML]{000000} 1:55:25}                           & {\color[HTML]{000000} 6,886}            & {\color[HTML]{000000} 367,437}        \\ \hline
{\color[HTML]{000000} 26}                             & {\color[HTML]{000000} 1:59:21}                          & {\color[HTML]{000000} 1:59:31}                           & {\color[HTML]{000000} 0,553}            & {\color[HTML]{000000} 33,473}         \\ \hline
\end{tabular}
\end{table}

% TABEL COLISSION
% Please add the following required packages to your document preamble:
% \usepackage[table,xcdraw]{xcolor}
% If you use beamer only pass "xcolor=table" option, i.e. \documentclass[xcolor=table]{beamer}
\begin{table}[]
\caption{Data Deteksi \textit{Colission}}
\label{tb:4_5}
\begin{tabular}{|c|c|c|c|}
\hline
\textit{No.} & {\color[HTML]{000000} \textit{ColissionStart}} & {\color[HTML]{000000} \textit{GameObjectTag}} & \textit{GameObjectName} \\ \hline
1            & {\color[HTML]{000000} 01.49.16 PM}             & {\color[HTML]{000000} Boundary}               & LeftLane                \\ \hline
2            & {\color[HTML]{000000} 01.49.18 PM}             & {\color[HTML]{000000} Boundary}               & LeftLane                \\ \hline
3            & {\color[HTML]{000000} 01.49.59 PM}             & {\color[HTML]{000000} Boundary}               & LeftLane                \\ \hline
4            & {\color[HTML]{000000} 01.50.13 PM}             & {\color[HTML]{000000} Boundary}               & RightLane               \\ \hline
5            & {\color[HTML]{000000} 01.50.39 PM}             & {\color[HTML]{000000} Boundary}               & RightLane               \\ \hline
6            & 01.51.39 PM                                    & OtherVehicle                                  & SportsVehicleYellow     \\ \hline
7            & 01.53.21 PM                                    & OtherVehicle                                  & SportsVehicleYellow     \\ \hline
8            & 01.59.11 PM                                    & OtherVehicle                                  & SportsVehicleYellow     \\ \hline
9            & 02.30.15 PM                                    & Boundary                                      & RightLane               \\ \hline
10           & 02.30.48 PM                                    & Boundary                                      & RightLane               \\ \hline
\end{tabular}
\end{table}


\cleardoublepage
\chapter{PENUTUP}
\vspace{1ex}

\section{Kesimpulan}
\vspace{1ex}

Dari hasil pengujian yang sudah dilakukan dapat ditarik beberapa kesimpulan sebagai berikut:
\vspace{1ex}

\begin{enumerate}[nolistsep]

\item Tombol - tombol jumlah lajur pada \textit{Interface - Main Menu} telah berkorelasi dengan benar terhadap \textit{scene} yang dimuat
\item Pengujian pengambilan data kecepatan menghasilkan data berdasarkan kalkulasi vektor global, diperlukan pengujian untuk memverifikasi keakuratan data tersebut.
\item Pengujian pengambilan data spasial menghasilkan data relatif posisi mobil terhadap garis pinggir jalan, dapat disimpulkan data tersebut dapat digunakan untuk suplemen data pengujian pengambilan data \textit{response time}, diperlukan pengujian untuk memverifikasi keakuratan data tersebut 
\item Pengujian citra webcam memiliki \textit{performance cost} yang sangat tinggi, yaitu \textit{execution time} tiap framenya mencapai 250-550 milisekon, hal ini disebabkan oleh proses unity dalam melakukan \textit{encoding} data berupa \textit{texture} menjadi suatu citra. Permasalahan ini ada pada level perangkat keras (GPU dan CPU). Diperlukannya suatu kompromi antara \textit{performance} dan akurasi
\item Proses kalkulasi data serta berjalannya \textit{script} utama pada \textit{unity} tidak terlalu berpengaruh terhadap respon \textit{steering wheel}, nilai error mendekati 0 persen atau akurat hingga 5 angka dibelakang koma (0.000001\%)(gambar \ref{fig:4.8}) hal ini disebabkan oleh kecilnya \textit{performance cost} dari \textit{script} tersebut (20-100 milisekon).
\item Pengujian UX tidak konklusif, yang disebabkan oleh situasi dan kondisi pandemi \textit{COVID-19}. Diperlukannya pengujian UX dengan jumlah responden yang lebih banyak sehingga dapat mewakili target demografi pengguna yang dituju.

\end{enumerate}
\vspace{1ex}

\section{Saran}
\vspace{1ex}

Untuk pengembangan penelitian selanjutnya terdapat beberapa saran sebagai berikut :
\vspace{1ex}

\begin{enumerate}[nolistsep]
	
	\item Melakukan \textit{refactor} / penataan ulang terhadap struktur source code.
	\vspace{1ex}
	
	\item Mengurangi \textit{performance cost} dari source code.
	\vspace{1ex}
	
	\item Meningkatkan estetik dari simulator mulai dari UI, kualitas objek 3D, serta animasi - animasi atau detail - detail  lain yang dapat meningkatkan imersifitas dari simulator.
	
	\item Menambah kapabilitas dari simulator dengan menambah jenis data yang bisa diambil oleh simulator.
	\vspace{1ex}

    \item Melakukan survey terhadap pengguna untuk fitur yang perlu ditambahkan pada simulator ini
	\vspace{1ex}
	
	\item Melakukan survey kuesioner dengan jumlah responden yang lebih banyak agar mewakili target demografi pengguna yang dituju
	\vspace{1ex}

\end{enumerate}
\cleardoublepage

% Daftar pustaka
\renewcommand*\bibname{DAFTAR PUSTAKA}
\addcontentsline{toc}{chapter}{\bibname}
\titlespacing*{\chapter}{0pt}{-4ex}{2ex}
\appendix
\bibliographystyle{ieeetr}
\bibliography{TA}
\cleardoublepage

% Biografi penulis
\begin{center}
\Large\textbf{BIOGRAFI PENULIS}
\end{center}
\vspace{1ex}

\begin{wrapfigure}{L}{0.3\textwidth}
	\centering
	\vspace{-3ex}	
	\includegraphics[width=0.31\textwidth]{img/Foto.jpg}
	\vspace{-4ex}
\end{wrapfigure}
\noindent
Penulis adalah salah satu mahasiswa S1 Departemen Teknik Komputer Fakultas Teknologi Elektro Institut Teknologi Sepuluh Nopember ITS. Penulis sangat tertarik dengan riset - riset yang berhubungan dengan sistem tertanam \textit{(embedded system)}, grafika komputer \textit{(computer graphics)}, dan visi komputer \textit{(Computer Vision)}.
\addcontentsline{toc}{chapter}{Biografi Penulis}
\cleardoublepage

\end{document}