\chapter{PENDAHULUAN}
\pagenumbering{arabic}
\vspace{1ex}

\section*{}
Penelitian ini di latar belakangi oleh berbagai kondisi yang menjadi acuan. Selain itu juga terdapat beberapa permasalahan yang akan dijawab sebagai luaran dari penelitian.
\vspace{1ex}

\section{Latar belakang}
\vspace{1ex}

Indonesia mulai memasuki periode aging population, dimana terjadi peningkatan umur harapan hidup yang diikuti dengan peningkatan jumlah lansia, dan Indonesia mengalami peningkatan jumlah penduduk lansia dari 18 juta jiwa (7,56\%) pada tahun 2010, menjadi 25,9 juta jiwa (9,7\%) pada tahun 2019, dan diperkirakan akan terus meningkat dimana tahun 2035 menjadi 48,2 juta jiwa (15,77\%) \cite{cit:1}. Berdasarkan data dari Badan Pusat Statistika (BPS) Indonesia keberadaan lansia yang tinggal sendiri, di mana persentasenya mencapai 9,38 \%. Jika dilihat berdasarkan tipe daerah, persentase lansia di perdesaan yang tinggal sendiri lebih tinggi dibandingkan lansia di perkotaan (10,10 \% berbanding 8,74 \%). Bahkan, terdapat kesenjangan yang cukup tinggi pada lansia yang tinggal sendiri antara lansia perempuan dengan lakilaki (13,39 \% berbanding 4,98 \%) \cite{cit:2}.Lansia yang tinggal sendiri digambarkan sebagai kelompok yang berisiko dan membutuhkan perhatian khusus \cite{cit:4}. 
\vspace{1ex}

Menurut data World Health Organization (WHO) pada tahun 2018, Jatuh adalah penyebab utama kedua kematian akibat cedera yang tidak disengaja atau tidak disengaja di seluruh dunia \cite{cit:5}. Berdasarkan data dari Centers for Disease Control and Prevention tingkat kematian akibat jatuh yang disesuaikan dengan usia adalah 64 kematian per 100.000 orang dewasa yang lebih tua, tingkat kematian akibat jatuh di antara orang dewasa berusia 65 tahun ke atas meningkat sekitar 30\% dari 2009 hingga 2018. \cite{cit:6}. Jatuh adalah penanda kelemahan, imobilitas, dan gangguan kesehatan akut dan kronis pada orang tua. Jatuh pada gilirannya mengurangi fungsinya dengan menyebabkan cedera, keterbatasan aktivitas, takut jatuh, dan kehilangan mobilitas. Kebanyakan cedera pada lansia adalah akibat jatuh; patah tulang pinggul, lengan bawah, humerus, dan panggul biasanya diakibatkan oleh efek gabungan dari jatuh dan osteoporosis \cite{cit:7}. Bahkan ketika cederanya tidak begitu serius, lansia sering kali kesulitan untuk bangkit tanpa bantuan \cite{cit:8},terkadang mengarah ke 'long-lie' di mana lansia tetap terjebak di lantai untuk waktu periode waktu yang lama. 'long-lie' dapat menyebabkan dehidrasi, \textit{Ulkus dekubitus}, pneumonia, hipotermia dan kematian \cite{cit:9}.
 \vspace{1ex} 

Teknologi untuk melakukan deteksi jatuh sudah ada, dan umumnya dapat dikategorikan menjadi 2 jenis, yaitu secara visual dan secara \textit{wearable}. Tetapi, lansia memiliki kecenderungan akan membawa hal penting sehingga deteksi secara \textit{wearable} tidak disarankan dikarenakan lansia harus selalu menggunakan alat tersebut yang juga membuat lansia tidak nyaman. Mayoritas lansia juga memiliki kondisi keuangan yang tidak kuat dikarenakan bergantung pada uang pensiun dan tidak memiliki pemasukan.
\vspace{1ex} 

Oleh karena itu, pada tugas akhir ini akan dikembangkan sebuah sistem untuk melakukan deteksi jatuh dan pelaporan berbasis visual dan juga sistem tertanam dengan harga terjangkau. Metode yang digunakan, yaitu 3D-CNN, berfungsi untuk melakukan deteksi jatuh yang melakukan deteksi secara \textit{frame sequence}, sehingga meningkatkan akurasi dan mengurangi \textit{false alarm}. Diharapkan dengan pengembangan tugas akhir ini, sistem dapat melakukan deteksi jatuh pada lansia sehingga dapat menghubungi keluarga terdekat jika terjadi kejadian jatuh. 
\vspace{1ex} 

\section{Permasalahan}
\vspace{1ex}

Berdasarkan data yang telah dipaparkan di latar belakang, dapat dirumuskan beberapa rumusan masalah sebagai berikut yaitu meningkatnya angka manusia lanjut usia serta jumlah manusia lanjut usia tinggal sendiri yang relatif banyak, dimana sulit mendapatkan bantuan. Lalu risiko jatuh pada manusia lanjut usia semakin meningkat dengan bertambahnya umur, dikarenakan penurunan fisik. Kemudian manusia lanjut usia mengalami  penurunan  fisik, sehingga  jika  mengalami  kecelakaan,  luka  yang diderita dapat menyebabkan kematian jika tidak cepat ditangani.
\vspace{1ex}

\section{Tujuan}
\vspace{1ex}

Penelitian ini bertujuan untuk merancang sebuah perangkat yang mampu untuk mendeteksi orang yang jatuh secara otomatis. Perangkat tersebut akan menggunakan input video dari \textit{IP Camera} yang diakses oleh sebuah \textit{Single Board Computer} yang didalamnya telah ditanamkan sebuah program pengolahan citra digital. Jika terdeteksi orang jatuh maka gambar akan dikirimkan kepada rumah sakit terdekat dan meminta bantuan. Alarm juga akan menyala untuk mencari orang terdekat dan memberitahukan keluarga baik secara SMS ataupun media sosial.
\vspace{1ex}

\section{Batasan masalah}
\vspace{1ex}
Batasan masalah yang timbul dari permasalahan Tugas Akhir ini adalah:
\vspace{1ex}
\begin{enumerate}[nolistsep]
    \item Pengujian dilakukan dalam ruangan
	\cite{cit:4}
	\vspace{1ex}

	\item Orang yang akan jatuh hanya 1 orang
	\vspace{1ex}
	
	\item Kegiatan Uji adalah kegiatan pengambilan data berupa :
	    \begin{enumerate}
	        \item Korelasi User Interface dengan Lajur yang dimuat.
            \item Kecepatan Mobil
            \item Informasi Spasial Mobil
            \item Respon Waktu Pengendara \textit{(Response Time)}
            \item Citra Wajah Pengendara
            \item \textit{Serial Data} dari \textit{Microcontroller}
            \item Respon Sinyal dari \textit{Steering Wheel Controller} terhadap simulator
            \item Kuesioner \textit{User Experience} / UX Pengguna
	    \end{enumerate}
	\vspace{1ex}
	
	\item Kegiatan Uji Menggunakan dataset pribadi
	\vspace{1ex}
		 
\end{enumerate}
\vspace{1ex}

\section{Sistematika Penulisan}
\vspace{1ex}
Laporan penelitian Tugas akhir ini tersusun dalam sistematika dan terstruktur sehingga mudah dipahami dan dipelajari oleh pembaca maupun seseorang yang ingin melanjutkan penelitian ini. Alur sistematika penulisan laporan penelitian ini yaitu:
\vspace{1ex}

\begin{enumerate}[nolistsep]
	\item BAB I Pendahuluan

	Bab ini berisi uraian tentang latar belakang permasalahan, penegasan dan alasan pemilihan judul, sistematika laporan, tujuan dan metodologi penelitian.
	\vspace{1ex}

	\item BAB II Dasar Teori

	Pada bab ini berisi tentang uraian secara sistematis teori-teori yang berhubungan dengan permasalahan yang dibahas pada penelitian ini. Teori-teori ini digunakan sebagai dasar dalam penelitian, yaitu sistem simulator dan pengambilan data variabel - variabel uji.
	\vspace{1ex}

	\item BAB III Perancangan Sistem dan Impementasi

	Bab ini berisi tentang penjelasan-penjelasan terkait eksperimen yang akan dilakukan dan langkah-langkah pengolahan data hingga menghasilkan visualisasi. Guna mendukung eksperimen pada penelitian ini, digunakanlah blok diagram atau \textit{work flow} agar penjelasan sistem yang akan dibuat dapat terlihat dan mudah dibaca untuk implementasi pada pelaksanaan tugas akhir.
	\vspace{1ex}

	\item BAB IV Pengujian dan Analisa

	Bab ini menjelaskan tentang pengujian eksperimen yang dilakukan terhadap data dan analisanya. Beberapa teknik visualisasi akan ditunjukan hasilnya pada bab ini dan dilakukan analisa terhadap hasil visualisasi dan informasi yang didapat dari hasil mengamati visualisasi yang tersaji
	\vspace{1ex}

	\item BAB V Penutup

	Bab ini merupakan penutup yang berisi kesimpulan yang diambil dari penelitian dan pengujian yang telah dilakukan. Saran dan kritik yang membangun untuk pengembangkan lebih lanjut juga dituliskan pada bab ini.
\end{enumerate}
\vspace{1ex}

\section{Relevansi}
\begin{enumerate}
    \item Validasi Skala Kantuk Karolinska \mbox{\textit{(Karolinska Sleepiness Scale)}} / KSS dengan Variabel – Variabel EEG
    \par
    Peneliti pada riset tersebut bertujuan untuk melakukan validasi terhadap skala tingkat kantuk yang di proposisikan oleh Institut Karolinska Swedia, dengan variabel – variabel EEG. Peneliti memiliki landasan riset data korelasi antara variabel – variabel EEG  yang sudah valid dan terbukti. Maka selanjutnya peneliti ingin melakukan analisa korelasi antara variabel – variabel EEG dengan tingkat kantuk yang dimiliki oleh subjek riset. Kesimpulan dari riset tersebut menyatakan, skala tingkat kantuk karolinska memiliki korelasi yang kuat dengan variabel – variabel EEG \cite{cit:6}
    \vspace{1ex}
    
    \item Tingkat Kantuk Subjektif,  Simulasi kinerja Mengemudi menggunakan durasi kedipan mata.
    \par
    Riset ini memiliki tujuan untuk memproposisikan Tingkat Kantuk secara subjektif, data yang digunakan pada paper riset ini berupa lama durasi tingkat kedipan mata, dari banyak data yang diambil pada riset ini didapatkan korelasi antara durasi kedip mata dengan tingkat kantuk, namun kemudian peneliti menyimpulkan bahwa untuk dapat menentukan pengukuran Tingkat Kantuk secara Subjektif, harus dilakukan replikasi riset ini beberapa kali lagi \cite{cit:7}
    \vspace{1ex}
    
    \item Persepsi – Waktu Respon terhadap Bahaya tak terduga dijalan
    \par
    \textit{Research Question} yang diangkat oleh peneliti pada riset ini adalah “Seberapa lama toleransi waktu yang diperbolehkan untuk pengemudi bereaksi terhadap bahaya tak terduga dijalan?”, atau pada riset ini disebut dengan Perception Time (PR). Peneliti juga menyebutkan bahwa terlalu tinggi menilai angka PR dapat meningkatkan biaya konstruksi jalan. Sedangkan terlalu rendah menilai angka PR dapat menyebabkan meningkatnya bahaya dijalan bagi pengendara kendaraan bermotor di jalan. Setelah melakukan penelitian, peneliti pada riset ini menyimpulkan bahwa waktu response yang ideal bagi 95\% populasi yang mengemudi adalah 1,6 detik. Peneliti menambahkan, kesimpulan tersebut hanya berlaku pada situasi yang sama dengan kondisi pengujian. Apabila terdapat bahaya yang lebih mengintimidasi, hal tersebut dapat menghasilkan kemungkinan nilai PR yang berbeda. \cite{cit:8}
    \vspace{1ex}11

\end{enumerate}
\vspace{1ex}